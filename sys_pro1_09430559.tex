\documentclass[a4j,11pt]{jarticle}
% ファイル先頭から\begin{document}までの内容(プレアンブル)については,
% 教員からの指示がない限り, { } の中を書き換えるだけでよい.

% ToDo: 提出要領に従って,適切な余白を設定する
\usepackage[top=25truemm,  bottom=30truemm,
            left=25truemm, right=25truemm]{geometry}

% ToDo: 提出要領に従って,適切なタイトル・サブタイトルを設定する
\title{システムプログラミング1 \\
       期末レポート}

% ToDo: 自分自身の氏名と学生番号に書き換える
\author{氏名: 山田 敬汰 (Yamada,Keita) \\
        学生番号: 09430559}

% ToDo: 教員の指示に従って適切に書き換える
\date{出題日: 2019年?月??日 \\   %todo 正しい日付に置き換える
      提出日: 2019年?月??日 \\
      締切日: 2019年?月??日 \\}  % 注:最後の\\は不要に見えるが必要.

% ToDo: 図を入れる場合,以下の1行を有効にする
%\usepackage{graphicx}

\begin{document}
\maketitle

% 目次つきの表紙ページにする場合はコメントを外す
%{\footnotesize \tableofcontents \newpage}

% % 以下の7行は提出用のレポートでは必ず消すこと
% \textbf{\small※執筆上の注意:本書は空想上の課題に対するレポートの
%     執筆例である.章の構成と書くべき内容の参考として提示するもの
%     であるため,課題内容やプログラムの仕様などは,
%     実際の演習課題の指示に従って適切にまとめ直す必要がある.
%     途中まで文を書いて「・・・」によって省略している箇所があるが,
%     これに穴埋めをすることで提出できるレポートになるわけではない.
%     また,サンプルと同じ書き出しで文章を書く必要はない.}

%%%%%%%%%%%%%%%%%%%%%%%%%%%%%%%%%%%%%%%%%%%%%%%%%%%%%%%%%%%%%%%%
\section{概要}
%%%%%%%%%%%%%%%%%%%%%%%%%%%%%%%%%%%%%%%%%%%%%%%%%%%%%%%%%%%%%%%%

% 以下の4行は提出用のレポートでは必ず消すこと
% \textbf{\small※執筆上の注意:概要は多すぎず少なすぎずが重要である.
%     特に,次の3点について,執筆者の取り組みの概略が読者(=教員)に
%     伝わるようにしよう.(1) このレポートで取り組んだ課題の内容,
%     (2) 実験等によって得られた結果,(3) 結果に対しておこなった考察.\\}

% 本演習では,外部からの入力データを計算機で扱える内部形式に変換して格納し,
% それらを操作する方法について学習する.
% 具体的には,標準入力から与えられる名簿のCSVデータをC言語の構造体の配列に格納し,
% それらをソートして表示するプログラムを作成する.

% 与えられたプログラムの基本仕様と要件,および,本レポートにおける実装の概要を以下に述べる.

% また,本レポートでは以下の考察課題について考察をおこなった.

% \begin{enumerate}
% \setlength{\parskip}{2pt}\setlength{\itemsep}{2pt}%この1行で箇条書きの行間を調整している
%     \item 不足機能についての考察をおこなった.特に,・・・(サンプルのため省略)
%     \item エラー処理についての考察をおこなった.例えば,・・・(サンプルのため省略)
%     \item 構造体 \verb|struct profile| がメモリ中を占めるバイト数について確認をおこなった.
%           具体的には,\verb|sizeof|演算子を使用して・・・(サンプルのため中略)・・・確認をおこなった.
% \end{enumerate}

%%%%%%%%%%%%%%%%%%%%%%%%%%%%%%%%%%%%%%%%%%%%%%%%%%%%%%%%%%%%%%%%
\section{課題1-1}
%%%%%%%%%%%%%%%%%%%%%%%%%%%%%%%%%%%%%%%%%%%%%%%%%%%%%%%%%%%%%%%%

% 以下の5行は提出用のレポートでは必ず消すこと
% \textbf{\small ※執筆上の注意:講義中の説明などに基づいて計画した作成方針についてまとめる.
%     例えば,どういう手順で作成をおこなったのか?作成にあたって何を重視したのか?
%     この例は架空の講義内容に基づいて書かれている.実際の講義に合わせて内容や節構成を精査すること.
%     なお,コーディング中に考え直した細かい内容は,できる限り,
%     この章ではなく後の「作成過程における考察」でまとめること.\\}


% 表示(\verb|%Pn|)は\verb|printf|で各項目毎に表示すればよい.
% ただし,・・・であることに注意が必要である.
% また,実装中に・・・ということがわかったため,
% ・・・のように実装をすることにしている.
% この実装に関する方針決定の詳細は後のxxxx節で説明する.

% ソートは,C の標準関数である \verb|qsort()| を使用することにする.
% 構造体の各メンバー毎にソートをするために,7つの比較関数を用意する.
% 7つの比較関数へのポインタを要素とする配列を宣言することによって
% 項目毎のソートを見通しよく行えるようにする.

% (※サンプルのため省略)

%%%%%%%%%%%%%%%%%%%%%%%%%%%%%%%%%%%%%%%%%%%%%%%%%%%%%%%%%%%%%%%%
\section{課題1-2}
%%%%%%%%%%%%%%%%%%%%%%%%%%%%%%%%%%%%%%%%%%%%%%%%%%%%%%%%%%%%%%%%

% 以下の6行は提出用のレポートでは必ず消すこと
% \textbf{\small ※執筆上の注意:変数や数値は$\backslash$verbや\$\$
%     で囲って,適切な書体で記述することを忘れずに.
%     なお,このサンプルでは``わざと''一部の処理を省略している.
%     見た目の違いを確認して,自分のレポートでは処理を忘れないようにしよう.
%     また,この章はこのレポートサンプルの2章に基づいて書かれているが,
%     そもそも2章が架空の講義内容に基づいて書かれている点に注意すること.\\}


%%%%%%%%%%%%%%%%%%%%%%%%%%%%%%%%%%%%%%%%%%%%%%%%%%%%%%%%%%%%%%%%
\section{課題1-3}
%%%%%%%%%%%%%%%%%%%%%%%%%%%%%%%%%%%%%%%%%%%%%%%%%%%%%%%%%%%%%%%%

% 提出するレポートでは以下5行は必ず消すこと
% \textbf{\small ※執筆上の注意:この節はプログラムの使用法を説明す
%     る節である.最低限,起動の方法,入力の形式と方法,出力の読み方
%     を入れること.当然,実装したコマンドすべてを説明すべきであるが,
%     このサンプルのように説明に使う実行例が1つである必要はない.
%     なお,このサンプルは架空の課題であり,動作環境も架空である.\\}


%%%%%%%%%%%%%%%%%%%%%%%%%%%%%%%%%%%%%%%%%%%%%%%%%%%%%%%%%%%%%%%%
\section{課題1-4}
%%%%%%%%%%%%%%%%%%%%%%%%%%%%%%%%%%%%%%%%%%%%%%%%%%%%%%%%%%%%%%%%

%以下の3行は提出レポートでは不要なため消すこと.
% \textbf{\small ※執筆上の注意:ここでは,作成中に試行錯誤した内容,
%     例えば,「Aという実装ではなくBという実装にしたのはなぜか?」
%     などについて,バランスよくまとめる.\\}


% (※サンプルのため省略)

% %--------------------------------------------------------------%
% \subsection{・・・についての考察}
% %--------------------------------------------------------------%

% ・・・については・・・という方針にしたが,・・・という方針にすることも考えられる.
% 今回は・・・ということを考えたため,・・・とすることにした.
% ただし,もし・・・であるならば,・・・は・・・よりも・・・であるから,
% ・・・という実装方針とするほうがよいだろう.

% (※サンプルのため省略)


%%%%%%%%%%%%%%%%%%%%%%%%%%%%%%%%%%%%%%%%%%%%%%%%%%%%%%%%%%%%%%%%
\section{課題1-5}
%%%%%%%%%%%%%%%%%%%%%%%%%%%%%%%%%%%%%%%%%%%%%%%%%%%%%%%%%%%%%%%%

%以下の5行は提出レポートでは不要なため消すこと.
% \textbf{\small ※執筆上の注意:考察課題を中心にまとめる.
%     自分で考えた考察課題を書くことも強く推奨しているが,
%     「作成過程における考察」とは区別して書くこと.
%     「作成されたプログラムから考察できること」を求めている.
%     また,単なる感想で終わるような内容を書いてはいけない.\\}

%%%%%%%%%%%%%%%%%%%%%%%%%%%%%%%%%%%%%%%%%%%%%%%%%%%%%%%%%%%%%%%%
\section{感想}
%%%%%%%%%%%%%%%%%%%%%%%%%%%%%%%%%%%%%%%%%%%%%%%%%%%%%%%%%%%%%%%%
%--------------------------------------------------------------%

%%%%%%%%%%%%%%%%%%%%%%%%%%%%%%%%%%%%%%%%%%%%%%%%%%%%%%%%%%%%%%%%
\section{作成したプログラム}\label{sec:sourcecode}
%%%%%%%%%%%%%%%%%%%%%%%%%%%%%%%%%%%%%%%%%%%%%%%%%%%%%%%%%%%%%%%%
%
% 行番号付きのリストを挿入
%   cat -n ds-sample.c > ds-sample.txt
% としたものを貼付する.
% なお, fold コマンドを使うと指定行数で折り返すことができる.
% 詳しくは fold --help を実行してヘルプを読んでみるとよい.
%
{\fontsize{10pt}{11pt} \selectfont
\begin{verbatim}
\end{verbatim}
}
%% 注:行送りの変更は"指定箇所を含む段落”に効果があらわれる.
%%     fontsizeコマンドを用いて,行送りを変える場合は,
%%     その {...} の前後に空白行を入れ,段落を変えるようにすること.
%%     なお,行先頭がコメントから始まる行は空白行とは扱われない.

% 以下の3行は提出用のレポートでは必ず消すこと.
% \textbf{\small ※執筆上の注意:余白部分に文字がはみ出していないか,よく確認する.
%     例えば,\LaTeX によるコンパイル時のWarningメッセージを確認しよう.
%     \texttt{Overfull hbox}が出ていたら,はみ出している場所があるはずである.}

\end{document}
