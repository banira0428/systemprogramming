\documentclass[a4j,11pt]{jarticle}
% ファイル先頭から\begin{document}までの内容(プレアンブル)については,
% 教員からの指示がない限り, { } の中を書き換えるだけでよい.

% ToDo: 提出要領に従って,適切な余白を設定する
\usepackage[top=25truemm,  bottom=30truemm,
            left=25truemm, right=25truemm]{geometry}

% ToDo: 提出要領に従って,適切なタイトル・サブタイトルを設定する
\title{システムプログラミング1 \\
       期末レポート}

% ToDo: 自分自身の氏名と学生番号に書き換える
\author{氏名: 山田 敬汰 (Yamada,Keita) \\
        学生番号: 09430559}

% ToDo: 教員の指示に従って適切に書き換える
\date{出題日: 2019年?月??日 \\   %todo 正しい日付に置き換える
      提出日: 2019年?月??日 \\
      締切日: 2019年?月??日 \\}  % 注:最後の\\は不要に見えるが必要.

% ToDo: 図を入れる場合,以下の1行を有効にする
%\usepackage{graphicx}

\begin{document}
\maketitle

% 目次つきの表紙ページにする場合はコメントを外す
%{\footnotesize \tableofcontents \newpage}


%%%%%%%%%%%%%%%%%%%%%%%%%%%%%%%%%%%%%%%%%%%%%%%%%%%%%%%%%%%%%%%%
\section{概要}
%%%%%%%%%%%%%%%%%%%%%%%%%%%%%%%%%%%%%%%%%%%%%%%%%%%%%%%%%%%%%%%%

本演習では,C言語で書かれたプログラムを機械語に変換する際に必要不可欠となるアセンブラについて,
いくつかの演習課題を解くことを通じて学びを深めた.具体的には,入出力機能の実装やアセンブリ言語内に登場する用語の解説,
そして,C言語で記述されているプログラムの再現(関数の実装,素数を表示するプログラム及びその保存方法の改造),を行った.
以下に,今回の授業内で実践した5つの演習課題についての詳しい内容を記述する.

%%%%%%%%%%%%%%%%%%%%%%%%%%%%%%%%%%%%%%%%%%%%%%%%%%%%%%%%%%%%%%%%
\section{課題1-1}
%%%%%%%%%%%%%%%%%%%%%%%%%%%%%%%%%%%%%%%%%%%%%%%%%%%%%%%%%%%%%%%%

\subsection{課題内容}
教科書A.8節 「入力と出力」に示されている方法と, 
A.9節 最後「システムコール」に示されている方法のそれぞれで "Hello World" を表示せよ.
両者の方式を比較し考察せよ.

\subsection{作成したプログラム}

\begin{verbatim}
      1          .data
      2          .align 2
      3  msg:
      4          .asciiz "Hello World"
      5
      6          .text
      7          .align 2
      8  main:
      9          la      $a0, msg
     10          li      $v0, 4
     11          syscall
     12          j       $ra
     13          
\end{verbatim}

\subsection{考察}

両者を比較した時に,システムコールを利用しないプログラムでは,入出力処理を行う際にアドレスを固定値で指定しているのに対し,
システムコールを利用するプログラムでは,入出力処理はシステムコールによりOS側に一任されているので
プログラム内で固定値のアドレスが指定されていない,という違いがある.
この違いによって,システムコールを利用する方のプログラムは入出力装置についての情報を持っていなくても動作するので,
入出力機器が変更されてもプログラムを(疎結合になっている)という利点がある.
その他にも,OS側が許可した動作しかできないようになっているので(各種機器とプロセッサ間を仲介し,互いが直接アクセスしないようにしている),
アドレスを直接指定するよりもセキュアである.

%%%%%%%%%%%%%%%%%%%%%%%%%%%%%%%%%%%%%%%%%%%%%%%%%%%%%%%%%%%%%%%%
\section{課題1-2}
%%%%%%%%%%%%%%%%%%%%%%%%%%%%%%%%%%%%%%%%%%%%%%%%%%%%%%%%%%%%%%%%

\subsection{課題内容}
アセンブリ言語中で使用する .data, .text および .align とは何か解説せよ.
 下記コード中の 6行目の .data がない場合,どうなるかについて考察せよ.

\begin{verbatim}
       1:         .text
       2:         .align  2
       3: _print_message:
       4:         la      $a0, msg
       5:         li      $v0, 4
       6:         .data
       7:         .align  2
       8: msg:
       9:         .asciiz "Hello!!\n"
      10:         .text
      11:         syscall
      12:         j       $ra
      13: main:
      14:         subu    $sp, $sp, 24
      15:         sw      $ra, 16($sp)
      16:         jal     _print_message
      17:         lw      $ra, 16($sp)
      18:         addu    $sp, $sp, 24
      19:         j       $ra
\end{verbatim}

\subsection{解答}

.data,.text および .align とは「アセンブラ指令」と呼ばれるもので,主にプログラムの実行前に行われる
前処理を制御するものである.つまり,これらの記述は機械語としてメモリに変換される訳ではない.
その中でも,.data と .text はメモリ中のどこに機械語を配置するかを制御している.
具体的には .data はデータをデータセグメントに配置することを指示し,.text はデータをテキストセグメントに配置することを指示している.
何故,データセグメントとテキストセグメントを区別する必要があるのかというと,データセグメントの内容には読み込みと書き込みが両方行われるのに対し,
テキストセグメントの内容は読み込みしか行われない(値が不変)ので,同一プログラムを他のプロセスで実行する時にテキスト部分を共有することができたり,
読み込み専用領域にデータを置くことができる,というメリットがあるからである.

また .align はオプションとして整数$n$を指定し,データが$2^{n}$ずつの領域に確保される様に余白を取るための命令である.
何故余白を取る必要があるのかというと,MIPSは32ビット(4バイト)ずつデータをCPUとやり取りするため,余白を取って(この場合は$n = 2$)データを綺麗に整列させておくことで,
アクセス回数を減らすことが可能となり,システムの効率化に繋がるからである.

最後に,上記のコード中の6行目の .data がない場合にどうなるかというと,

%%%%%%%%%%%%%%%%%%%%%%%%%%%%%%%%%%%%%%%%%%%%%%%%%%%%%%%%%%%%%%%%
\section{課題1-3}
%%%%%%%%%%%%%%%%%%%%%%%%%%%%%%%%%%%%%%%%%%%%%%%%%%%%%%%%%%%%%%%%

\subsection{課題内容}
教科書A.6節 「手続き呼出し規約」に従って,関数 fact を実装せよ.
(以降の課題においては,この規約に全て従うこと) fact をC言語で記述した場合は,以下のようになるであろう.
\begin{verbatim}
      1: main()
      2: {
      3:   print_string("The factorial of 10 is ");
      4:   print_int(fact(10));
      5:   print_string("\n");
      6: }
      7: 
      8: int fact(int n)
      9: {
     10:   if (n < 1)
     11:     return 1;
     12:   else
     13:     return n * fact(n - 1);
     14: }
\end{verbatim}

%%%%%%%%%%%%%%%%%%%%%%%%%%%%%%%%%%%%%%%%%%%%%%%%%%%%%%%%%%%%%%%%
\section{課題1-4}
%%%%%%%%%%%%%%%%%%%%%%%%%%%%%%%%%%%%%%%%%%%%%%%%%%%%%%%%%%%%%%%%
\subsection{課題内容}
素数を最初から100番目まで求めて表示するMIPSのアセンブリ言語プログラムを作成してテストせよ. 
その際,素数を求めるために下記の2つのルーチンを作成すること.

\begin{itemize}
      \item test\_prime(n)    nが素数なら1,そうでなければ0を返す
      \item main()       整数を順々に素数判定し,100個プリント
\end{itemize}

\subsubsection{C言語で記述したプログラム例}

\begin{verbatim}
      1: int test_prime(int n)
      2: {
      3:   int i;
      4:   for (i = 2; i < n; i++){
      5:     if (n % i == 0)
      6:       return 0;
      7:   }
      8:   return 1;
      9: }
     10: 
     11: int main()
     12: {
     13:   int match = 0, n = 2;
     14:   while (match < 100){
     15:     if (test_prime(n) == 1){
     16:       print_int(n);
     17:       print_string(" ");
     18:       match++;
     19:     }
     20:     n++;
     21:   }
     22:   print_string("\n");
     23: }
\end{verbatim}

%%%%%%%%%%%%%%%%%%%%%%%%%%%%%%%%%%%%%%%%%%%%%%%%%%%%%%%%%%%%%%%%
\section{課題1-5}
%%%%%%%%%%%%%%%%%%%%%%%%%%%%%%%%%%%%%%%%%%%%%%%%%%%%%%%%%%%%%%%%

\subsection{課題内容}

素数を最初から100番目まで求めて表示するMIPSのアセンブリ言語プログラムを作成してテストせよ. 
ただし,配列に実行結果を保存するように main 部分を改造し, 
ユーザの入力によって任意の番目の配列要素を表示可能にせよ.

\subsubsection{C言語で記述したプログラム例}

\begin{verbatim}
      1: int primes[100];
      2: int main()
      3: {
      4:   int match = 0, n = 2;
      5:   while (match < 100){
      6:     if (test_prime(n) == 1){
      7:       primes[match++] = n;
      8:     }
      9:     n++;
     10:   }
     11:   for (;;){
     12:     print_string("> ");
     13:     print_int(primes[read_int() - 1]);
     14:     print_string("\n");
     15:   }
     16: }
\end{verbatim}

%%%%%%%%%%%%%%%%%%%%%%%%%%%%%%%%%%%%%%%%%%%%%%%%%%%%%%%%%%%%%%%%
\section{感想}
%%%%%%%%%%%%%%%%%%%%%%%%%%%%%%%%%%%%%%%%%%%%%%%%%%%%%%%%%%%%%%%%
%--------------------------------------------------------------%

%%%%%%%%%%%%%%%%%%%%%%%%%%%%%%%%%%%%%%%%%%%%%%%%%%%%%%%%%%%%%%%%
\section{作成したプログラム}\label{sec:sourcecode}
%%%%%%%%%%%%%%%%%%%%%%%%%%%%%%%%%%%%%%%%%%%%%%%%%%%%%%%%%%%%%%%%


{\fontsize{10pt}{11pt} \selectfont
\begin{verbatim}
\end{verbatim}
}

\end{document}
