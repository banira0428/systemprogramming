\documentclass[a4j,11pt]{jarticle}
% ファイル先頭から\begin{document}までの内容(プレアンブル)については,
% 教員からの指示がない限り, { } の中を書き換えるだけでよい.

% ToDo: 提出要領に従って,適切な余白を設定する
\usepackage[top=25truemm,  bottom=30truemm,
            left=25truemm, right=25truemm]{geometry}

% ToDo: 提出要領に従って,適切なタイトル・サブタイトルを設定する
\title{システムプログラミング1 \\
       期末レポート}

% ToDo: 自分自身の氏名と学生番号に書き換える
\author{氏名: 山田 敬汰 (Yamada,Keita) \\
        学生番号: 09430559}

% ToDo: 教員の指示に従って適切に書き換える
\date{出題日: 2019年?月??日 \\   %todo 正しい日付に置き換える
      提出日: 2019年?月??日 \\
      締切日: 2019年?月??日 \\}  % 注:最後の\\は不要に見えるが必要.

% ToDo: 図を入れる場合,以下の1行を有効にする
%\usepackage{graphicx}

\begin{document}
\maketitle

% 目次つきの表紙ページにする場合はコメントを外す
%{\footnotesize \tableofcontents \newpage}

% % 以下の7行は提出用のレポートでは必ず消すこと
% \textbf{\small※執筆上の注意:本書は空想上の課題に対するレポートの
%     執筆例である.章の構成と書くべき内容の参考として提示するもの
%     であるため,課題内容やプログラムの仕様などは,
%     実際の演習課題の指示に従って適切にまとめ直す必要がある.
%     途中まで文を書いて「・・・」によって省略している箇所があるが,
%     これに穴埋めをすることで提出できるレポートになるわけではない.
%     また,サンプルと同じ書き出しで文章を書く必要はない.}

%%%%%%%%%%%%%%%%%%%%%%%%%%%%%%%%%%%%%%%%%%%%%%%%%%%%%%%%%%%%%%%%
\section{概要}
%%%%%%%%%%%%%%%%%%%%%%%%%%%%%%%%%%%%%%%%%%%%%%%%%%%%%%%%%%%%%%%%

% 以下の4行は提出用のレポートでは必ず消すこと
% \textbf{\small※執筆上の注意:概要は多すぎず少なすぎずが重要である.
%     特に,次の3点について,執筆者の取り組みの概略が読者(=教員)に
%     伝わるようにしよう.(1) このレポートで取り組んだ課題の内容,
%     (2) 実験等によって得られた結果,(3) 結果に対しておこなった考察.\\}

本演習では,C言語で書かれたプログラムを機械語に変換する際に必要不可欠となるアセンブラについて,
いくつかの演習課題を解くことを通じて学びを深めた.具体的には,入出力機能の実装やアセンブリ言語内に登場する用語の解説,
そして,C言語で記述されているプログラムの再現(関数の実装,素数を表示するプログラム及びその保存方法の改造),を行った.
以下に,今回の授業内で実践した5つの演習課題についての詳しい内容を記述する.

% 本演習では,外部からの入力データを計算機で扱える内部形式に変換して格納し,
% それらを操作する方法について学習する.
% 具体的には,標準入力から与えられる名簿のCSVデータをC言語の構造体の配列に格納し,
% それらをソートして表示するプログラムを作成する.

% 与えられたプログラムの基本仕様と要件,および,本レポートにおける実装の概要を以下に述べる.

% また,本レポートでは以下の考察課題について考察をおこなった.

% \begin{enumerate}
% \setlength{\parskip}{2pt}\setlength{\itemsep}{2pt}%この1行で箇条書きの行間を調整している
%     \item 不足機能についての考察をおこなった.特に,・・・(サンプルのため省略)
%     \item エラー処理についての考察をおこなった.例えば,・・・(サンプルのため省略)
%     \item 構造体 \verb|struct profile| がメモリ中を占めるバイト数について確認をおこなった.
%           具体的には,\verb|sizeof|演算子を使用して・・・(サンプルのため中略)・・・確認をおこなった.
% \end{enumerate}

%%%%%%%%%%%%%%%%%%%%%%%%%%%%%%%%%%%%%%%%%%%%%%%%%%%%%%%%%%%%%%%%
\section{課題1-1}
%%%%%%%%%%%%%%%%%%%%%%%%%%%%%%%%%%%%%%%%%%%%%%%%%%%%%%%%%%%%%%%%

\subsection{課題内容}
教科書A.8節 「入力と出力」に示されている方法と, 
A.9節 最後「システムコール」に示されている方法のそれぞれで "Hello World" を表示せよ.
両者の方式を比較し考察せよ.

\subsection{作成したプログラム}

\begin{verbatim}
      
\end{verbatim}

\subsection{考察}

両者を比較した時に,システムコールを利用しないプログラムでは,入出力処理を行う際にアドレスを固定値で指定しているのに対し,
システムコールを利用するプログラムでは,入出力処理はシステムコールによりOS側に一任されているので
プログラム内で固定値のアドレスが指定されていない,という違いがある.
この違いによって,システムコールを利用する方のプログラムは入出力装置についての情報を持っていなくても動作するので,
入出力機器が変更されてもプログラムを(疎結合になっている)という利点がある.
その他にも,OS側が許可した動作しかできないようになっているので(各種機器とプロセッサ間を仲介し,互いが直接アクセスしないようにしている),
アドレスを直接指定するよりもセキュアである.


% 以下の5行は提出用のレポートでは必ず消すこと
% \textbf{\small ※執筆上の注意:講義中の説明などに基づいて計画した作成方針についてまとめる.
%     例えば,どういう手順で作成をおこなったのか?作成にあたって何を重視したのか?
%     この例は架空の講義内容に基づいて書かれている.実際の講義に合わせて内容や節構成を精査すること.
%     なお,コーディング中に考え直した細かい内容は,できる限り,
%     この章ではなく後の「作成過程における考察」でまとめること.\\}


% 表示(\verb|%Pn|)は\verb|printf|で各項目毎に表示すればよい.
% ただし,・・・であることに注意が必要である.
% また,実装中に・・・ということがわかったため,
% ・・・のように実装をすることにしている.
% この実装に関する方針決定の詳細は後のxxxx節で説明する.

% ソートは,C の標準関数である \verb|qsort()| を使用することにする.
% 構造体の各メンバー毎にソートをするために,7つの比較関数を用意する.
% 7つの比較関数へのポインタを要素とする配列を宣言することによって
% 項目毎のソートを見通しよく行えるようにする.

% (※サンプルのため省略)

%%%%%%%%%%%%%%%%%%%%%%%%%%%%%%%%%%%%%%%%%%%%%%%%%%%%%%%%%%%%%%%%
\section{課題1-2}
%%%%%%%%%%%%%%%%%%%%%%%%%%%%%%%%%%%%%%%%%%%%%%%%%%%%%%%%%%%%%%%%

\subsection{課題内容}
アセンブリ言語中で使用する .data, .text および .align とは何か解説せよ.
 下記コード中の 6行目の .data がない場合,どうなるかについて考察せよ.

\begin{verbatim}
      1:         .text
      2:         .align  2
      3: _print_message:
      4:         la      $a0, msg
      5:         li      $v0, 4
      6:         .data
      7:         .align  2
      8: msg:
      9:         .asciiz "Hello!!\n"
      10:         .text
      11:         syscall
      12:         j       $ra
      13: main:
      14:         subu    $sp, $sp, 24
      15:         sw      $ra, 16($sp)
      16:         jal     _print_message
      17:         lw      $ra, 16($sp)
      18:         addu    $sp, $sp, 24
      19:         j       $ra
\end{verbatim}

\subsection{解答}

.data, .text および .align とは「アセンブラ指令」と呼ばれるもので,主にプログラムの実行前に行われる
前処理を制御するものである.つまり,これらの記述は機械語としてメモリに変換される訳ではない.その中でも,


% 以下の6行は提出用のレポートでは必ず消すこと
% \textbf{\small ※執筆上の注意:変数や数値は$\backslash$verbや\$\$
%     で囲って,適切な書体で記述することを忘れずに.
%     なお,このサンプルでは``わざと''一部の処理を省略している.
%     見た目の違いを確認して,自分のレポートでは処理を忘れないようにしよう.
%     また,この章はこのレポートサンプルの2章に基づいて書かれているが,
%     そもそも2章が架空の講義内容に基づいて書かれている点に注意すること.\\}


%%%%%%%%%%%%%%%%%%%%%%%%%%%%%%%%%%%%%%%%%%%%%%%%%%%%%%%%%%%%%%%%
\section{課題1-3}
%%%%%%%%%%%%%%%%%%%%%%%%%%%%%%%%%%%%%%%%%%%%%%%%%%%%%%%%%%%%%%%%

\subsection{課題内容}
教科書A.6節 「手続き呼出し規約」に従って,関数 fact を実装せよ.
(以降の課題においては,この規約に全て従うこと) fact をC言語で記述した場合は,以下のようになるであろう.
\begin{verbatim}
      1: main()
      2: {
      3:   print_string("The factorial of 10 is ");
      4:   print_int(fact(10));
      5:   print_string("\n");
      6: }
      7: 
      8: int fact(int n)
      9: {
     10:   if (n < 1)
     11:     return 1;
     12:   else
     13:     return n * fact(n - 1);
     14: }
\end{verbatim}


% 提出するレポートでは以下5行は必ず消すこと
% \textbf{\small ※執筆上の注意:この節はプログラムの使用法を説明す
%     る節である.最低限,起動の方法,入力の形式と方法,出力の読み方
%     を入れること.当然,実装したコマンドすべてを説明すべきであるが,
%     このサンプルのように説明に使う実行例が1つである必要はない.
%     なお,このサンプルは架空の課題であり,動作環境も架空である.\\}


%%%%%%%%%%%%%%%%%%%%%%%%%%%%%%%%%%%%%%%%%%%%%%%%%%%%%%%%%%%%%%%%
\section{課題1-4}
%%%%%%%%%%%%%%%%%%%%%%%%%%%%%%%%%%%%%%%%%%%%%%%%%%%%%%%%%%%%%%%%
\subsection{課題内容}
素数を最初から100番目まで求めて表示するMIPSのアセンブリ言語プログラムを作成してテストせよ. 
その際,素数を求めるために下記の2つのルーチンを作成すること.

\begin{itemize}
      \item test\_prime(n)    nが素数なら1,そうでなければ0を返す
      \item main()       整数を順々に素数判定し,100個プリント
\end{itemize}

\subsubsection{C言語で記述したプログラム例}

\begin{verbatim}
      1: int test_prime(int n)
      2: {
      3:   int i;
      4:   for (i = 2; i < n; i++){
      5:     if (n % i == 0)
      6:       return 0;
      7:   }
      8:   return 1;
      9: }
     10: 
     11: int main()
     12: {
     13:   int match = 0, n = 2;
     14:   while (match < 100){
     15:     if (test_prime(n) == 1){
     16:       print_int(n);
     17:       print_string(" ");
     18:       match++;
     19:     }
     20:     n++;
     21:   }
     22:   print_string("\n");
     23: }
\end{verbatim}


%以下の3行は提出レポートでは不要なため消すこと.
% \textbf{\small ※執筆上の注意:ここでは,作成中に試行錯誤した内容,
%     例えば,「Aという実装ではなくBという実装にしたのはなぜか?」
%     などについて,バランスよくまとめる.\\}


% (※サンプルのため省略)

% %--------------------------------------------------------------%
% \subsection{・・・についての考察}
% %--------------------------------------------------------------%

% ・・・については・・・という方針にしたが,・・・という方針にすることも考えられる.
% 今回は・・・ということを考えたため,・・・とすることにした.
% ただし,もし・・・であるならば,・・・は・・・よりも・・・であるから,
% ・・・という実装方針とするほうがよいだろう.

% (※サンプルのため省略)


%%%%%%%%%%%%%%%%%%%%%%%%%%%%%%%%%%%%%%%%%%%%%%%%%%%%%%%%%%%%%%%%
\section{課題1-5}
%%%%%%%%%%%%%%%%%%%%%%%%%%%%%%%%%%%%%%%%%%%%%%%%%%%%%%%%%%%%%%%%

\subsection{課題内容}

素数を最初から100番目まで求めて表示するMIPSのアセンブリ言語プログラムを作成してテストせよ. 
ただし,配列に実行結果を保存するように main 部分を改造し, 
ユーザの入力によって任意の番目の配列要素を表示可能にせよ.

\subsubsection{C言語で記述したプログラム例}

\begin{verbatim}
      1: int primes[100];
      2: int main()
      3: {
      4:   int match = 0, n = 2;
      5:   while (match < 100){
      6:     if (test_prime(n) == 1){
      7:       primes[match++] = n;
      8:     }
      9:     n++;
     10:   }
     11:   for (;;){
     12:     print_string("> ");
     13:     print_int(primes[read_int() - 1]);
     14:     print_string("\n");
     15:   }
     16: }
\end{verbatim}

%以下の5行は提出レポートでは不要なため消すこと.
% \textbf{\small ※執筆上の注意:考察課題を中心にまとめる.
%     自分で考えた考察課題を書くことも強く推奨しているが,
%     「作成過程における考察」とは区別して書くこと.
%     「作成されたプログラムから考察できること」を求めている.
%     また,単なる感想で終わるような内容を書いてはいけない.\\}

%%%%%%%%%%%%%%%%%%%%%%%%%%%%%%%%%%%%%%%%%%%%%%%%%%%%%%%%%%%%%%%%
\section{感想}
%%%%%%%%%%%%%%%%%%%%%%%%%%%%%%%%%%%%%%%%%%%%%%%%%%%%%%%%%%%%%%%%
%--------------------------------------------------------------%

%%%%%%%%%%%%%%%%%%%%%%%%%%%%%%%%%%%%%%%%%%%%%%%%%%%%%%%%%%%%%%%%
\section{作成したプログラム}\label{sec:sourcecode}
%%%%%%%%%%%%%%%%%%%%%%%%%%%%%%%%%%%%%%%%%%%%%%%%%%%%%%%%%%%%%%%%
%
% 行番号付きのリストを挿入
%   cat -n ds-sample.c > ds-sample.txt
% としたものを貼付する.
% なお, fold コマンドを使うと指定行数で折り返すことができる.
% 詳しくは fold --help を実行してヘルプを読んでみるとよい.
%
{\fontsize{10pt}{11pt} \selectfont
\begin{verbatim}
\end{verbatim}
}
%% 注:行送りの変更は"指定箇所を含む段落”に効果があらわれる.
%%     fontsizeコマンドを用いて,行送りを変える場合は,
%%     その {...} の前後に空白行を入れ,段落を変えるようにすること.
%%     なお,行先頭がコメントから始まる行は空白行とは扱われない.

% 以下の3行は提出用のレポートでは必ず消すこと.
% \textbf{\small ※執筆上の注意:余白部分に文字がはみ出していないか,よく確認する.
%     例えば,\LaTeX によるコンパイル時のWarningメッセージを確認しよう.
%     \texttt{Overfull hbox}が出ていたら,はみ出している場所があるはずである.}

\end{document}
