\documentclass[a4j,11pt]{jarticle}
% ファイル先頭から\begin{document}までの内容(プレアンブル)については,
% 教員からの指示がない限り, { } の中を書き換えるだけでよい.

% ToDo: 提出要領に従って,適切な余白を設定する
\usepackage[top=25truemm,  bottom=30truemm,
            left=25truemm, right=25truemm]{geometry}

% ToDo: 提出要領に従って,適切なタイトル・サブタイトルを設定する
\title{システムプログラミング2 \\
       期末レポート}

% ToDo: 自分自身の氏名と学生番号に書き換える
\author{氏名: 山田 敬汰 (Yamada,Keita) \\
        学生番号: 09430559}

% ToDo: 教員の指示に従って適切に書き換える
\date{出題日: 2019年12月9日 \\   %todo 正しい日付に置き換える
      提出日: 2019年??月??日 \\
      締切日: 2020年1月27日 \\}  % 注:最後の\\は不要に見えるが必要.

% ToDo: 図を入れる場合,以下の1行を有効にする
%\usepackage{graphicx}

\begin{document}
\maketitle

% 目次つきの表紙ページにする場合はコメントを外す
%{\footnotesize \tableofcontents \newpage}

%%%%%%%%%%%%%%%%%%%%%%%%%%%%%%%%%%%%%%%%%%%%%%%%%%%%%%%%%%%%%%%%
\section{概要}
%%%%%%%%%%%%%%%%%%%%%%%%%%%%%%%%%%%%%%%%%%%%%%%%%%%%%%%%%%%%%%%%

本演習では,前回の演習で学習したアセンブリ言語について,システムコールを用いたC言語との連携,
C言語のプログラムをコンパイルした際の手続き呼び出し規約に基づいたスタックの取り扱い,
{\tt auto}変数と{\tt static}変数という宣言方法によって異なる二種類の変数のライフサイクル,
そしてアセンブリ言語での可変引数関数の実現についての考察および実装等,
これまでよりもより深い部分について理解を深めた.
以下に,今回の授業内で実践した5つの演習課題についての詳しい内容を記述する.

%%%%%%%%%%%%%%%%%%%%%%%%%%%%%%%%%%%%%%%%%%%%%%%%%%%%%%%%%%%%%%%%
\section{課題2-1}
%%%%%%%%%%%%%%%%%%%%%%%%%%%%%%%%%%%%%%%%%%%%%%%%%%%%%%%%%%%%%%%%
\subsection{課題内容}
SPIMが提供するシステムコールを C言語から実行できるようにしたい. 教科書A.6節 「手続き呼出し規約」に従って,各種手続きをアセンブラで記述せよ. ファイル名は,{\tt syscalls.s} とすること.

また,記述した {\tt syscalls.s} の関数をC言語から呼び出すことで, ハノイの塔({\tt hanoi.c} とする)を完成させよ.

\subsection{C言語で記述したプログラム例}

\begin{verbatim}
      1	#include <stdio.h>
      2	
      3	void hanoi(int n, int start, int finish, int extra)
      4	{
      5	    if (n != 0)
      6	    {
      7	        hanoi(n - 1, start, extra, finish);
      8	        print_string("Move disk ");
      9	        print_int(n);
     10	        print_string(" from peg ");
     11	        print_int(start);
     12	        print_string(" to peg ");
     13	        print_int(finish);
     14	        print_string(".\n");
     15	        hanoi(n - 1, extra, finish, start);
     16	    }
     17	}
     18	main()
     19	{
     20	    int n;
     21	    print_string("Enter number of disks> ");
     22	    n = read_int();
     23	    hanoi(n, 1, 2, 3);
     24	}     
\end{verbatim}

\subsection{作成したプログラム}
\begin{verbatim}
      1	.text
      2	.align 2
      3	
      4	_print_int:
      5	 subu    $sp,$sp,24
      6	 sw      $ra,20($sp)
      7	
      8	 li      $v0, 1
      9	 syscall
     10	
     11	 lw      $ra,20($sp)
     12	 addu    $sp,$sp,24
     13	 j       $ra 
     14	
     15	_print_string:
     16	 subu    $sp,$sp,24
     17	 sw      $ra,20($sp)
     18	
     19	 li      $v0, 4
     20	 syscall
     21	
     22	 lw      $ra,20($sp)
     23	 addu    $sp,$sp,24
     24	 j       $ra 
     25	
     26	_read_int:
     27	 subu    $sp,$sp,24
     28	 sw      $ra,20($sp)
     29	
     30	 li      $v0, 5
     31	 syscall
     32	
     33	 lw      $ra,20($sp)
     34	 addu    $sp,$sp,24
     35	 j       $ra 
     36	
     37	_read_string:
     38	 subu    $sp,$sp,24
     39	 sw      $ra,20($sp)
     40	
     41	 li      $v0, 8
     42	 syscall
     43	
     44	 lw      $ra,20($sp)
     45	 addu    $sp,$sp,24
     46	 j       $ra
     47	
     48	_exit:
     49	 li      $v0, 10
     50	 syscall
     51	
\end{verbatim}

\subsection{実行結果}

\begin{verbatim}
      Enter number of disks>  4
      Move disk 1 from peg 1 to peg 3.
      Move disk 2 from peg 1 to peg 2.
      Move disk 1 from peg 3 to peg 2.
      Move disk 3 from peg 1 to peg 3.
      Move disk 1 from peg 2 to peg 1.
      Move disk 2 from peg 2 to peg 3.
      Move disk 1 from peg 1 to peg 3.
      Move disk 4 from peg 1 to peg 2.
      Move disk 1 from peg 3 to peg 2.
      Move disk 2 from peg 3 to peg 1.
      Move disk 1 from peg 2 to peg 1.
      Move disk 3 from peg 3 to peg 2.
      Move disk 1 from peg 1 to peg 3.
      Move disk 2 from peg 1 to peg 2.
      Move disk 1 from peg 3 to peg 2.           
\end{verbatim}

\subsection{プログラムの解説}
ここでは{\tt \_print\_int}部分での手続きを例に解説する.
\begin{enumerate}
      \item スタックの領域を$24$バイト確保し,戻りアドレスを格納しておく.戻りアドレスを退避させている理由としては,
      {\tt syscall}内の手続きがOSに一任され,アセンブリのプログラムから分からないようになっているからである.
      (つまり,OSが勝手に$\$ra$レジスタの値を壊している可能性を考慮している.)
      \item $\$v0$レジスタに適切な番号({\tt \_print\_int}の場合は$1$)を格納し,{\tt syscall}命令を発行する.
      \item スタックに格納しておいた戻りアドレスを$\$ra$レジスタに再び格納し,呼び出し元に帰る.
\end{enumerate}

その他の手続きも$\$v0$レジスタの値を変更することで,同様の手順で実行することができる.(OSによって抽象化されている.)

また,{\tt exit}手続きのみ,スタックに戻りアドレスを格納していない.
その理由としては{\tt syscall}の発行によってプロセスが終了するので,
値を退避させたところで復帰させる方法がないからである.
ただ,手続き呼び出しを厳密に守る場合はスタックへの退避の手続きを記述する場合もある.その場合でも実行結果は変わらない.

\subsection{考察}

今回のプログラムでは{\tt syscall}を呼び出す部分のみをアセンブリ言語で記述している.
これは,C言語の中から直接{\tt syscall}命令を発行する方法が存在しないからである.
そして,実行する時にC言語の部分をアセンブリ言語にコンパイルし,{\tt syscalls.s}と共に正しい順序でメモリ上に読み込むことで,
プログラムを実行することができる.ここで,ファイル読み込みを間違えた場合は関数の参照先が未定義となり,プログラムが例外を発生する.

また,手続き呼び出し規約によって「引数はどのレジスタに入っていて,戻り値はこのレジスタに入っている」というのが決められているので,
この規約を守っている限りはC言語とアセンブリ言語との連携を円滑に行うことができる.
言い換えれば,プログラマはコンパイラがどのようにC言語のプログラムを変換するのか(レジスタを決定するルール)を知っていなければアセンブリ言語を書くことができない,ということである.

%%%%%%%%%%%%%%%%%%%%%%%%%%%%%%%%%%%%%%%%%%%%%%%%%%%%%%%%%%%%%%%%
\section{課題2-2}
%%%%%%%%%%%%%%%%%%%%%%%%%%%%%%%%%%%%%%%%%%%%%%%%%%%%%%%%%%%%%%%%
\subsection{課題内容}
{\tt hanoi.s} を例に {\tt spim-gcc} の引数保存に関するスタックの利用方法について,説明せよ. 
そのことは,規約上許されるスタックフレームの最小値$24$とどう関係しているか. 
このスタックフレームの最小値規約を守らないとどのような問題が生じるかについて解説せよ.

\subsection{与えられたプログラム}
\begin{verbatim}
      1	      .file	1 "hanoi.c"
      2	
            (中略)
     13	
     14	
     15	     .rdata
     16	     .align	0
     17	     .align	2
     18	$LC0:
     19	     .ascii	"Move disk \000"
     20	     .align	2
     21	$LC1:
     22	     .ascii	" from peg \000"
     23	     .align	2
     24	$LC2:
     25	     .ascii	" to peg \000"
     26	     .align	2
     27	$LC3:
     28	     .ascii	".\n\000"
     29	     .text
     30	     .align	2
     31	     .set	nomips16
     32	_hanoi:
     33	     subu	$sp,$sp,24
     34	     sw	$ra,20($sp)
     35	     sw	$fp,16($sp)
     36	     move	$fp,$sp
     37	     sw	$a0,24($fp)
     38	     sw	$a1,28($fp)
     39	     sw	$a2,32($fp)
     40	     sw	$a3,36($fp)
     41	     lw	$v0,24($fp)
     42	     beq	$v0,$zero,$L3
     43	     lw	$v0,24($fp)
     44	     addu	$v0,$v0,-1
     45	     move	$a0,$v0
     46	     lw	$a1,28($fp)
     47	     lw	$a2,36($fp)
     48	     lw	$a3,32($fp)
     49	     jal	_hanoi
     50	     la	$a0,$LC0
     51	     jal	_print_string
     52	     lw	$a0,24($fp)
     53	     jal	_print_int
     54	     la	$a0,$LC1
     55	     jal	_print_string
     56	     lw	$a0,28($fp)
     57	     jal	_print_int
     58	     la	$a0,$LC2
     59	     jal	_print_string
     60	     lw	$a0,32($fp)
     61	     jal	_print_int
     62	     la	$a0,$LC3
     63	     jal	_print_string
     64	     lw	$v0,24($fp)
     65	     addu	$v0,$v0,-1
     66	     move	$a0,$v0
     67	     lw	$a1,36($fp)
     68	     lw	$a2,32($fp)
     69	     lw	$a3,28($fp)
     70	     jal	_hanoi
     71	$L3:
     72	     move	$sp,$fp
     73	     lw	$ra,20($sp)
     74	     lw	$fp,16($sp)
     75	     addu	$sp,$sp,24
     76	     j	$ra
     77	     .rdata
     78	     .align	0
     79	     .align	2
     80	$LC4:
     81	     .ascii	"Enter number of disks> \000"
     82	     .text
     83	     .align	2
     84	     .set	nomips16
     85	main:
     86	     subu	$sp,$sp,32
     87	     sw	$ra,28($sp)
     88	     sw	$fp,24($sp)
     89	     move	$fp,$sp
     90	     la	$a0,$LC4
     91	     jal	_print_string
     92	     jal	_read_int
     93	     sw	$v0,16($fp)
     94	     lw	$a0,16($fp)
     95	     li	$a1,1			# 0x1
     96	     li	$a2,2			# 0x2
     97	     li	$a3,3			# 0x3
     98	     jal	_hanoi
     99	     move	$sp,$fp
    100	     lw	$ra,28($sp)
    101	     lw	$fp,24($sp)
    102	     addu	$sp,$sp,32
    103	     j	$ra
\end{verbatim}

\subsection{解説}

{\tt spim-gcc}の引数保存について,表\ref{tbl:2-2}に示す.
具体的な保存手順について説明すると,以下のようになる.
\begin{enumerate}
      \item \label{itm:main}スタックを32バイト確保し,$\$fp$,$\$ra$ レジスタの値を退避させる.(82-85行目)
      \item \label{itm:hanoi_begin}{\tt hanoi}ラベルにジャンプした後,スタックをさらに24バイト確保し,$\$fp$,$\$ra$ レジスタの値を退避させる.(31-33行目)
      \item \label{itm:hanoi_end}\ref{itm:main}で確保したスタックの空き領域に,($\$a0-\$a3$)を退避させる.(呼び出された側で退避の判断を行うことで,余計な書き込み処理を減らすことが可能となる)(35-38行目)
      \item {\tt hanoi}ラベルに再帰的にジャンプするので,必要な回数だけ\ref{itm:hanoi_begin}-\ref{itm:hanoi_end}間の処理を繰り返す.
\end{enumerate}

\begin{table}[t]
      \centering
      \caption{スタックを用いた引数保存}
      \label{tbl:2-2}
      \begin{tabular}{|l|l|l|l|l|}
      \hline
      \$sp       & offset & 内容   & 備考       & 行番号(格納処理の場所) \\ \hline
:          & :      & :    & :        & :            \\ \hline
31行目での\$sp & -56    & \$a0 & 第1引数     & 35行目(2回目)    \\ \hline
           & -52    & \$a1 & 第2引数     & 36行目(2回目)    \\ \hline
           & -48    & \$a2 & 第3引数     & 37行目(2回目)    \\ \hline
           & -44    & \$a3 & 第4引数     & 38行目(2回目)    \\ \hline
           & -40    & \$fp & フレームポインタ & 33行目         \\ \hline
           & -36    & \$ra & 戻りアドレス   & 32行目         \\ \hline
82行目での\$sp & -32    & \$a0 & 第1引数     & 35行目         \\ \hline
           & -28    & \$a1 & 第2引数     & 36行目         \\ \hline
           & -24    & \$a2 & 第3引数     & 37行目         \\ \hline
           & -20    & \$a3 & 第4引数     & 38行目         \\ \hline
           & -16    & \$v0 & 戻り値      & 89行目         \\ \hline
           & -12    &      &          &              \\ \hline
           & -8     & \$fp & フレームポインタ & 84行目         \\ \hline
           & -4     & \$ra & 戻りアドレス   & 83行目         \\ \hline
      \end{tabular}
      \end{table}




また,規約上許されてるスタックフレームの最小値が$24$となっているのは,被呼び出し関数が退避させる可能性のある
4つの引数レジスタ($\$a0-\$a3$),呼び出し関数の戻りアドレス($\$ra$),そして呼び出し関数のフレームポインタ($\$fp$)
のための領域を確保する必要があるからである(レジスタ6個分).

このスタックフレームの最小値について,守らない場合どうなるかについて以下に解説する.
\begin{itemize}
      \item 呼び出された側が守らない場合:自身が確保した領域内に引数レジスタ($\$a0-\$a3$)を退避するので,不具合は発生しない.
      (呼び出し側が無駄な空き領域を確保しただけで済む)
      \item 呼び出した側が守らない場合:呼び出された側は,呼び出し側が引数レジスタ($\$a0-\$a3$)
      を保存するためのスタック領域を確保してくれている前提で引数レジスタ($\$a0-\$a3$)を退避させようとするので,
      呼び出し側がスタックに格納している値を破壊してしまい,不具合が発生する.
\end{itemize}

%%%%%%%%%%%%%%%%%%%%%%%%%%%%%%%%%%%%%%%%%%%%%%%%%%%%%%%%%%%%%%%%
\section{課題2-3}
%%%%%%%%%%%%%%%%%%%%%%%%%%%%%%%%%%%%%%%%%%%%%%%%%%%%%%%%%%%%%%%%
\subsection{課題内容}
以下のプログラム {\tt report2-1.c} をコンパイルした結果をもとに, {\tt auto}変数と{\tt static}変数の違い,ポインタと配列の違いについてレポートせよ.

\subsection{与えられたプログラム}
\begin{verbatim}
      1	int primes_stat[10];
      2	
      3	char *string_ptr = "ABCDEFG";
      4	char string_ary[] = "ABCDEFG";
      5	
      6	void print_var(char *name, int val)
      7	{
      8	    print_string(name);
      9	    print_string(" = ");
     10	    print_int(val);
     11	    print_string("\n");
     12	}
     13	
     14	main()
     15	{
     16	    int primes_auto[10];
     17	
     18	    primes_stat[0] = 2;
     19	    primes_auto[0] = 3;
     20	
     21	    print_var("primes_stat[0]", primes_stat[0]);
     22	    print_var("primes_auto[0]", primes_auto[0]);
     23	}
\end{verbatim}

\subsection{与えられたプログラム(コンパイル後)}\label{sec:2-3}
\begin{verbatim}
      1	    .file	1 "2-3.c"
      2	
            (中略)
     13	
     14	
     15	    .rdata
     16	    .align	0
     17	    .align	2
     18	$LC0:
     19	    .ascii	"ABCDEFG\000"
     20	    .data
     21	    .align	0
     22	    .align	2
     23	_string_ptr:
     24	    .word	$LC0
     25	    .align	2
     26	_string_ary:
     27	    .ascii	"ABCDEFG\000"
     28	    .rdata
     29	    .align	0
     30	    .align	2
     31	$LC1:
     32	    .ascii	" = \000"
     33	    .align	2
     34	$LC2:
     35	    .ascii	"\n\000"
     36	    .text
     37	    .align	2
     38	    .set	nomips16
     39	_print_var:
     40	    subu	$sp,$sp,24
     41	    sw	$ra,20($sp)
     42	    sw	$fp,16($sp)
     43	    move	$fp,$sp
     44	    sw	$a0,24($fp)
     45	    sw	$a1,28($fp)
     46	    lw	$a0,24($fp)
     47	    jal	_print_string
     48	    la	$a0,$LC1
     49	    jal	_print_string
     50	    lw	$a0,28($fp)
     51	    jal	_print_int
     52	    la	$a0,$LC2
     53	    jal	_print_string
     54	    move	$sp,$fp
     55	    lw	$ra,20($sp)
     56	    lw	$fp,16($sp)
     57	    addu	$sp,$sp,24
     58	    j	$ra
     59	    .rdata
     60	    .align	0
     61	    .align	2
     62	$LC3:
     63	    .ascii	"primes_stat[0]\000"
     64	    .align	2
     65	$LC4:
     66	    .ascii	"primes_auto[0]\000"
     67	    .text
     68	    .align	2
     69	    .set	nomips16
     70	main:
     71	    subu	$sp,$sp,64
     72	    sw	$ra,60($sp)
     73	    sw	$fp,56($sp)
     74	    move	$fp,$sp
     75	    li	$v0,2			# 0x2
     76	    sw	$v0,_primes_stat
     77	    li	$v0,3			# 0x3
     78	    sw	$v0,16($fp)
     79	    la	$a0,$LC3
     80	    lw	$a1,_primes_stat
     81	    jal	_print_var
     82	    la	$a0,$LC4
     83	    lw	$a1,16($fp)
     84	    jal	_print_var
     85	    move	$sp,$fp
     86	    lw	$ra,60($sp)
     87	    lw	$fp,56($sp)
     88	    addu	$sp,$sp,64
     89	    j	$ra
     90	
     91	    .comm	_primes_stat,40
\end{verbatim}

\subsection{考察}

まず,{\tt auto}変数と{\tt static}変数の違いについて考察する.
{\tt auto}変数とは,関数内で動的に確保され,その関数が終了するまで値を保持し続ける変数である.今回与えられたプログラム内では{\tt primes\_auto}が該当する.
アセンブリコード内を見ると{\tt main}ラベルの直後で64バイトのスタックが確保されており,
{\tt primes\_auto}の値を保存するためにスタック領域が利用されているということが分かる.ここで,確保するスタックのサイズが64バイトとなっているのは,
手続き呼び出し規約で定められている最低限のスタックサイズ24バイトと{\tt primes\_auto}の値を格納するために必要な領域($4 × 10 = 40$バイト)の和が64バイトだからである.
これを確かめるために,簡易的なC言語のプログラムを作成した({\tt auto}変数に関する処理のみで構成されたプログラム).

\begin{verbatim}
      1	main()
      2	{
      3	    int primes_auto[2];
      4	
      5	    primes_auto[0] = 0;
      6	    primes_auto[1] = 1;
      7	
      8	    print_int(primes_auto[0]);
      9	    print_int(primes_auto[1]);
     10	}
\end{verbatim}

これをコンパイルすると以下のようなコードが生成される.

\begin{verbatim}
      1    		    .file	1 "2-3-2.c"
      2    	
                    (中略)
     13    	
     14    	
     15    		    .text
     16    		    .align	2
     17    	    main:
     18    		    subu	$sp,$sp,32        # 24 + 4 * 2
     19    		    sw	$ra,28($sp)
     20    		    sw	$fp,24($sp)
     21    		    move	$fp,$sp
     22    		    sw	$zero,16($fp)
     23    		    li	$v0,1			# 0x1
     24    		    sw	$v0,20($fp)
     25    		    lw	$a0,16($fp)
     26    		    jal	_print_int
     27    		    lw	$a0,20($fp)
     28    		    jal	_print_int
     29    		    move	$sp,$fp
     30    		    lw	$ra,28($sp)
     31    		    lw	$fp,24($sp)
     32    		    addu	$sp,$sp,32
     33    		    j	$ra
\end{verbatim}

このコンパイル結果からも分かるように,関数内で変数を宣言した場合は,関数の開始時に{\tt auto}変数の値を確保できるようにスタックを多めに確保している.
そして,変数の値はスタックに格納され,処理が終わるとともに値は失われてしまう.(スタックが解放される)

これに対して,{\tt static}変数とは,プログラムの開始時に静的に確保され,プログラムの終了時まで値を保持し続ける変数である.
今回与えられたプログラムでは{\tt primes\_stat}がこれに該当する.
アセンブリコード内を見ると,91行目のアセンブリ指令{\tt .comm}の部分でメモリ内に40バイト分の領域を確保しており,
{\tt primes\_stat}の値を保存するためにこの領域が使用されていることが分かる.
この領域はアセンブリ指令で確保された場所であり,プログラム終了まで解放されることはない(つまり,プログラムの終了時まで値を保持することができる).

この二種類の変数のメリットとデメリットについて比較した結果を表\ref{tbl:2-3}にまとめる.

\begin{table}[]
      \centering
      \caption{二種類の変数の比較}
      \label{tbl:2-3}
      \begin{tabular}{|l|l|l|}
            \hline 
                  & {\tt auto変数}                                                                                                                                                    & {\tt static変数}                                                                                                                     \\ \hline
            保存場所  & メモリ(スタック部)                                                                                                                                                & メモリ(データ部)                                                                                                                    \\ \hline
            メリット  & \begin{tabular}[c]{@{}l@{}}関数の中でのみ値を保持しているので\\ メモリ空間の節約ができる.\\ (使うときだけ確保する)\\ スタックに積み上げながら値を保持している\\ ので,再帰呼び出しで書き込みが\\ 起きた際でも,別々の場所に値を格納できる.\end{tabular} & \begin{tabular}[c]{@{}l@{}}メモリ内の固定した位置に値を\\ 格納しているので,再利用が容易.\\ スタックを使う場合と比較して,\\ 読み書きを素早く行うことができる.\\ (確保,解放が不要)\end{tabular} \\ \hline
            デメリット & \begin{tabular}[c]{@{}l@{}}同じ値を何度も呼び出す場合だと,\\ スタックの確保および解放処理を\\ 無駄に実行する事になる.\end{tabular}                                                                 & \begin{tabular}[c]{@{}l@{}}値を格納する位置は一箇所に固定\\ されているため,再帰呼び出しを\\ してしまうと意図せず値を\\ 上書きしてしまう危険性がある.\end{tabular}                   \\ \hline
            \end{tabular}
      \end{table}

次に,配列とポインタの違いについて考察する.C言語でこれらを用いる際には,変数に格納されているのはどちらも先頭アドレスでなので,
ほぼ同様に扱えるが,コンパイル後のアセンブリコードを見ると,明確な相違点が存在する.
それは,配列として宣言している場合は該当ラベルの位置に値がそのまま書かれている(\ref{sec:2-3}節のプログラムの27行目)のに対し,
ポインタとして宣言している場合は該当ラベルの位置に直接値が書かれているのではなく,その値が書かれているラベルが書かれており(\ref{sec:2-3}節のプログラムの24行目),
間接的に値にアクセスしているということである.

この二つを比較するために作成した簡易的なC言語のプログラムを以下に示す.

\begin{verbatim}
      1  char *string_ptr = "ABCDEFG";
      2  char string_ary[] = "ABCDEFG";
      3
      4  main()
      5  {
      6      string_ptr = "AB";
      7      //string_ary = "AB"; コンパイルエラー発生
      8  }
\end{verbatim}

これをコンパイルすると以下のようなコードが生成される.

\begin{verbatim}
      1          .file   1 "2-3-3.c"
      2
                  (中略)
     13
     14
     15          .rdata
     16          .align  2
     17  $LC0:
     18          .asciiz "ABCDEFG"
     19          .data
     20          .align  2
     21  _string_ptr:
     22          .word   $LC0
     23          .align  2
     24  _string_ary:
     25          .asciiz "ABCDEFG"
     26          .rdata
     27          .align  2
     28  $LC1:
     29          .asciiz "AB"
     30          .text
     31          .align  2
     32  main:
     33          subu    $sp,$sp,8
     34          sw      $fp,0($sp)
     35          move    $fp,$sp
     36          la      $v0,$LC1
     37          sw      $v0,_string_ptr
     38          move    $sp,$fp
     39          lw      $fp,0($sp)
     40          addu    $sp,$sp,8
     41          j       $ra
\end{verbatim}

上記のプログラムから分かるように,配列とポインタの大きな違いとしては,宣言した変数に再び代入処理ができるかどうか,ということが挙げられる.
配列の場合では,変数内に文字列の先頭アドレスが定数として保存されているので,別の値を代入することはできずコンパイルエラーが発生する.(添字付きでアクセスするのは問題ない)
これに対してポインタの場合では,変数内に格納されているアドレスを自由に付け替えれる様になっているので,再代入が可能であり,それによって元の値が壊されることもない.
(アクセスする場所が変わっているだけ.)ただ,ポインタを使った場合では使わなくなった値を適切に解放する処理をしなければ,
どこからもアクセスされないデータが蓄積し,メモリを圧迫する場合もあるので注意しなければならない.

%%%%%%%%%%%%%%%%%%%%%%%%%%%%%%%%%%%%%%%%%%%%%%%%%%%%%%%%%%%%%%%%
\section{課題2-4}
%%%%%%%%%%%%%%%%%%%%%%%%%%%%%%%%%%%%%%%%%%%%%%%%%%%%%%%%%%%%%%%%
\subsection{課題内容}

\subsection{可変引数}


\subsection{可変引数(C言語)}

\subsection{アセンブリでの実現方法の違い(Cとの)}

\subsection{実装}

%%%%%%%%%%%%%%%%%%%%%%%%%%%%%%%%%%%%%%%%%%%%%%%%%%%%%%%%%%%%%%%%
\section{課題2-5}
%%%%%%%%%%%%%%%%%%%%%%%%%%%%%%%%%%%%%%%%%%%%%%%%%%%%%%%%%%%%%%%%

\subsection{課題内容}

%%%%%%%%%%%%%%%%%%%%%%%%%%%%%%%%%%%%%%%%%%%%%%%%%%%%%%%%%%%%%%%%
\section{感想}

\end{document}
