\documentclass[a4j,11pt]{jarticle}
% ファイル先頭から\begin{document}までの内容(プレアンブル)については,
% 教員からの指示がない限り, { } の中を書き換えるだけでよい.

% ToDo: 提出要領に従って,適切な余白を設定する
\usepackage[top=25truemm,  bottom=30truemm,
            left=25truemm, right=25truemm]{geometry}

% ToDo: 提出要領に従って,適切なタイトル・サブタイトルを設定する
\title{システムプログラミング2 \\
       期末レポート}

% ToDo: 自分自身の氏名と学生番号に書き換える
\author{氏名: 山田 敬汰 (Yamada,Keita) \\
        学生番号: 09430559}

% ToDo: 教員の指示に従って適切に書き換える
\date{出題日: 2019年12月9日 \\   %todo 正しい日付に置き換える
      提出日: 2019年1月20日 \\
      締切日: 2020年1月27日 \\}  % 注:最後の\\は不要に見えるが必要.

% ToDo: 図を入れる場合,以下の1行を有効にする
%\usepackage{graphicx}

\begin{document}
\maketitle

% 目次つきの表紙ページにする場合はコメントを外す
%{\footnotesize \tableofcontents \newpage}

%%%%%%%%%%%%%%%%%%%%%%%%%%%%%%%%%%%%%%%%%%%%%%%%%%%%%%%%%%%%%%%%
\section{概要}
%%%%%%%%%%%%%%%%%%%%%%%%%%%%%%%%%%%%%%%%%%%%%%%%%%%%%%%%%%%%%%%%

本演習では,前回の演習で学習したアセンブリ言語について,システムコールを用いたC言語との連携,
C言語のプログラムをコンパイルした際の手続き呼び出し規約に基づいたスタックの取り扱い,
{\tt auto}変数と{\tt static}変数という宣言方法によって異なる二種類の変数のライフサイクル,
そしてアセンブリ言語での可変引数関数の実現についての考察および実装等,
これまでよりもより深い部分について理解を深めた.
以下に,今回の授業内で実践した5つの演習課題についての詳しい内容を記述する.

%%%%%%%%%%%%%%%%%%%%%%%%%%%%%%%%%%%%%%%%%%%%%%%%%%%%%%%%%%%%%%%%
\section{課題2-1}
%%%%%%%%%%%%%%%%%%%%%%%%%%%%%%%%%%%%%%%%%%%%%%%%%%%%%%%%%%%%%%%%
\subsection{課題内容}
SPIMが提供するシステムコールを C言語から実行できるようにしたい. 教科書A.6節 「手続き呼出し規約」に従って,各種手続きをアセンブラで記述せよ. ファイル名は,{\tt syscalls.s} とすること.

また,記述した {\tt syscalls.s} の関数をC言語から呼び出すことで, ハノイの塔({\tt hanoi.c} とする)を完成させよ.

\subsection{C言語で記述したプログラム例}

\begin{verbatim}
      1	#include <stdio.h>
      2	
      3	void hanoi(int n, int start, int finish, int extra)
      4	{
      5	    if (n != 0)
      6	    {
      7	        hanoi(n - 1, start, extra, finish);
      8	        print_string("Move disk ");
      9	        print_int(n);
     10	        print_string(" from peg ");
     11	        print_int(start);
     12	        print_string(" to peg ");
     13	        print_int(finish);
     14	        print_string(".\n");
     15	        hanoi(n - 1, extra, finish, start);
     16	    }
     17	}
     18	main()
     19	{
     20	    int n;
     21	    print_string("Enter number of disks> ");
     22	    n = read_int();
     23	    hanoi(n, 1, 2, 3);
     24	}     
\end{verbatim}

\subsection{実行結果}
\begin{verbatim}
      Enter number of disks>  4
      Move disk 1 from peg 1 to peg 3.
      Move disk 2 from peg 1 to peg 2.
      Move disk 1 from peg 3 to peg 2.
      Move disk 3 from peg 1 to peg 3.
      Move disk 1 from peg 2 to peg 1.
      Move disk 2 from peg 2 to peg 3.
      Move disk 1 from peg 1 to peg 3.
      Move disk 4 from peg 1 to peg 2.
      Move disk 1 from peg 3 to peg 2.
      Move disk 2 from peg 3 to peg 1.
      Move disk 1 from peg 2 to peg 1.
      Move disk 3 from peg 3 to peg 2.
      Move disk 1 from peg 1 to peg 3.
      Move disk 2 from peg 1 to peg 2.
      Move disk 1 from peg 3 to peg 2.           
\end{verbatim}

\subsection{プログラムの解説}
ここでは{\tt \_print\_int}部分での手続きを例に解説する.
\begin{enumerate}
      \item スタックの領域を$24$バイト確保し,戻りアドレスを格納しておく.戻りアドレスを退避させている理由としては,
      {\tt syscall}内の手続きがOSに一任され,アセンブリのプログラムから分からないようになっているからである.
      (つまり,OSが勝手に$\$ra$レジスタの値を壊している可能性を考慮している.)
      \item $\$v0$レジスタに適切な番号({\tt \_print\_int}の場合は$1$)を格納し,{\tt syscall}命令を発行する.
      \item スタックに格納しておいた戻りアドレスを$\$ra$レジスタに再び格納し,呼び出し元に帰る.
\end{enumerate}

その他の手続きも$\$v0$レジスタの値を変更することで,同様の手順で実行することができる.(OSによって抽象化されている.)

また,{\tt exit}手続きのみ,スタックに戻りアドレスを格納していない.
その理由としては{\tt syscall}の発行によってプロセスが終了するので,
値を退避させたところで復帰させる方法がないからである.
ただ,手続き呼び出しを厳密に守る場合はスタックへの退避の手続きを記述する場合もある.その場合でも実行結果は変わらない.

\subsection{考察}

今回のプログラムでは{\tt syscall}を呼び出す部分のみをアセンブリ言語で記述している.
これは,C言語の中から直接{\tt syscall}命令を発行する方法が存在しないからである.
そして,実行する時にC言語の部分をアセンブリ言語にコンパイルし,{\tt syscalls.s}と共に正しい順序でメモリ上に読み込むことで,
プログラムを実行することができる.ここで,ファイル読み込みを間違えた場合は関数の参照先が未定義となり,プログラムが例外を発生する.

また,手続き呼び出し規約によって「引数はどのレジスタに入っていて,戻り値はこのレジスタに入っている」というのが決められているので,
この規約を守っている限りはC言語とアセンブリ言語との連携を円滑に行うことができる.
言い換えれば,プログラマはコンパイラがどのようにC言語のプログラムを変換するのか(レジスタを決定するルール)を知っていなければアセンブリ言語を書くことができない,ということである.

\subsection{作成したプログラム}
\begin{verbatim}
      1	.text
      2	.align 2
      3	
      4	_print_int:
      5	 subu    $sp,$sp,24
      6	 sw      $ra,20($sp)
      7	
      8	 li      $v0, 1
      9	 syscall
     10	
     11	 lw      $ra,20($sp)
     12	 addu    $sp,$sp,24
     13	 j       $ra 
     14	
     15	_print_string:
     16	 subu    $sp,$sp,24
     17	 sw      $ra,20($sp)
     18	
     19	 li      $v0, 4
     20	 syscall
     21	
     22	 lw      $ra,20($sp)
     23	 addu    $sp,$sp,24
     24	 j       $ra 
     25	
     26	_read_int:
     27	 subu    $sp,$sp,24
     28	 sw      $ra,20($sp)
     29	
     30	 li      $v0, 5
     31	 syscall
     32	
     33	 lw      $ra,20($sp)
     34	 addu    $sp,$sp,24
     35	 j       $ra 
     36	
     37	_read_string:
     38	 subu    $sp,$sp,24
     39	 sw      $ra,20($sp)
     40	
     41	 li      $v0, 8
     42	 syscall
     43	
     44	 lw      $ra,20($sp)
     45	 addu    $sp,$sp,24
     46	 j       $ra
     47	
     48	_exit:
     49	 li      $v0, 10
     50	 syscall
     51	
     52	 	
\end{verbatim}

%%%%%%%%%%%%%%%%%%%%%%%%%%%%%%%%%%%%%%%%%%%%%%%%%%%%%%%%%%%%%%%%
\section{課題2-2}
%%%%%%%%%%%%%%%%%%%%%%%%%%%%%%%%%%%%%%%%%%%%%%%%%%%%%%%%%%%%%%%%
\subsection{課題内容}
{\tt hanoi.s} を例に {\tt spim-gcc} の引数保存に関するスタックの利用方法について,説明せよ. 
そのことは,規約上許されるスタックフレームの最小値$24$とどう関係しているか. 
このスタックフレームの最小値規約を守らないとどのような問題が生じるかについて解説せよ.

\subsection{与えられたプログラム}
\begin{verbatim}
      1	      .file	1 "hanoi.c"
      2	
            (中略)
     13	
     14	
     15	     .rdata
     16	     .align	0
     17	     .align	2
     18	$LC0:
     19	     .ascii	"Move disk \000"
     20	     .align	2
     21	$LC1:
     22	     .ascii	" from peg \000"
     23	     .align	2
     24	$LC2:
     25	     .ascii	" to peg \000"
     26	     .align	2
     27	$LC3:
     28	     .ascii	".\n\000"
     29	     .text
     30	     .align	2
     31	     .set	nomips16
     32	_hanoi:
     33	     subu	$sp,$sp,24
     34	     sw	$ra,20($sp)
     35	     sw	$fp,16($sp)
     36	     move	$fp,$sp
     37	     sw	$a0,24($fp)
     38	     sw	$a1,28($fp)
     39	     sw	$a2,32($fp)
     40	     sw	$a3,36($fp)
     41	     lw	$v0,24($fp)
     42	     beq	$v0,$zero,$L3
     43	     lw	$v0,24($fp)
     44	     addu	$v0,$v0,-1
     45	     move	$a0,$v0
     46	     lw	$a1,28($fp)
     47	     lw	$a2,36($fp)
     48	     lw	$a3,32($fp)
     49	     jal	_hanoi
     50	     la	$a0,$LC0
     51	     jal	_print_string
     52	     lw	$a0,24($fp)
     53	     jal	_print_int
     54	     la	$a0,$LC1
     55	     jal	_print_string
     56	     lw	$a0,28($fp)
     57	     jal	_print_int
     58	     la	$a0,$LC2
     59	     jal	_print_string
     60	     lw	$a0,32($fp)
     61	     jal	_print_int
     62	     la	$a0,$LC3
     63	     jal	_print_string
     64	     lw	$v0,24($fp)
     65	     addu	$v0,$v0,-1
     66	     move	$a0,$v0
     67	     lw	$a1,36($fp)
     68	     lw	$a2,32($fp)
     69	     lw	$a3,28($fp)
     70	     jal	_hanoi
     71	$L3:
     72	     move	$sp,$fp
     73	     lw	$ra,20($sp)
     74	     lw	$fp,16($sp)
     75	     addu	$sp,$sp,24
     76	     j	$ra
     77	     .rdata
     78	     .align	0
     79	     .align	2
     80	$LC4:
     81	     .ascii	"Enter number of disks> \000"
     82	     .text
     83	     .align	2
     84	     .set	nomips16
     85	main:
     86	     subu	$sp,$sp,32
     87	     sw	$ra,28($sp)
     88	     sw	$fp,24($sp)
     89	     move	$fp,$sp
     90	     la	$a0,$LC4
     91	     jal	_print_string
     92	     jal	_read_int
     93	     sw	$v0,16($fp)
     94	     lw	$a0,16($fp)
     95	     li	$a1,1			# 0x1
     96	     li	$a2,2			# 0x2
     97	     li	$a3,3			# 0x3
     98	     jal	_hanoi
     99	     move	$sp,$fp
    100	     lw	$ra,28($sp)
    101	     lw	$fp,24($sp)
    102	     addu	$sp,$sp,32
    103	     j	$ra
\end{verbatim}

\subsection{解説}

{\tt spim-gcc}の引数保存について,表\ref{tbl:2-2}に示す.
具体的な保存手順について説明すると,以下のようになる.
\begin{enumerate}
      \item \label{itm:main}スタックを32バイト確保し,$\$fp$,$\$ra$ レジスタの値を退避させる.(82-85行目)
      \item \label{itm:hanoi_begin}{\tt hanoi}ラベルにジャンプした後,スタックをさらに24バイト確保し,$\$fp$,$\$ra$ レジスタの値を退避させる.(31-33行目)
      \item \label{itm:hanoi_end}\ref{itm:main}で確保したスタックの空き領域に,($\$a0-\$a3$)を退避させる.(呼び出された側で退避の判断を行うことで,余計な書き込み処理を減らすことが可能となる)(35-38行目)
      \item {\tt hanoi}ラベルに再帰的にジャンプするので,必要な回数だけ\ref{itm:hanoi_begin}-\ref{itm:hanoi_end}間の処理を繰り返す.
\end{enumerate}

\begin{table}[t]
      \centering
      \caption{スタックを用いた引数保存}
      \label{tbl:2-2}
      \begin{tabular}{|l|l|l|l|l|}
      \hline
      \$sp       & offset & 内容   & 備考       & 行番号(格納処理の場所) \\ \hline
:          & :      & :    & :        & :            \\ \hline
31行目での\$sp & -56    & \$a0 & 第1引数     & 35行目(2回目)    \\ \hline
           & -52    & \$a1 & 第2引数     & 36行目(2回目)    \\ \hline
           & -48    & \$a2 & 第3引数     & 37行目(2回目)    \\ \hline
           & -44    & \$a3 & 第4引数     & 38行目(2回目)    \\ \hline
           & -40    & \$fp & フレームポインタ & 33行目         \\ \hline
           & -36    & \$ra & 戻りアドレス   & 32行目         \\ \hline
82行目での\$sp & -32    & \$a0 & 第1引数     & 35行目         \\ \hline
           & -28    & \$a1 & 第2引数     & 36行目         \\ \hline
           & -24    & \$a2 & 第3引数     & 37行目         \\ \hline
           & -20    & \$a3 & 第4引数     & 38行目         \\ \hline
           & -16    & \$v0 & 戻り値      & 89行目         \\ \hline
           & -12    &      &          &              \\ \hline
           & -8     & \$fp & フレームポインタ & 84行目         \\ \hline
           & -4     & \$ra & 戻りアドレス   & 83行目         \\ \hline
      \end{tabular}
      \end{table}




また,規約上許されてるスタックフレームの最小値が$24$となっているのは,被呼び出し関数が退避させる可能性のある
4つの引数レジスタ($\$a0-\$a3$),呼び出し関数の戻りアドレス($\$ra$),そして呼び出し関数のフレームポインタ($\$fp$)
のための領域を確保する必要があるからである(レジスタ6個分).

このスタックフレームの最小値について,守らない場合どうなるかについて以下に解説する.
\begin{itemize}
      \item 呼び出された側が守らない場合:自身が確保した領域内に引数レジスタ($\$a0-\$a3$)を退避するので,不具合は発生しない.
      (呼び出し側が無駄な空き領域を確保しただけで済む)
      \item 呼び出した側が守らない場合:呼び出された側は,呼び出し側が引数レジスタ($\$a0-\$a3$)
      を保存するためのスタック領域を確保してくれている前提で引数レジスタ($\$a0-\$a3$)を退避させようとするので,
      呼び出し側がスタックに格納している値を破壊してしまい,不具合が発生する.
\end{itemize}

%%%%%%%%%%%%%%%%%%%%%%%%%%%%%%%%%%%%%%%%%%%%%%%%%%%%%%%%%%%%%%%%
\section{課題2-3}
%%%%%%%%%%%%%%%%%%%%%%%%%%%%%%%%%%%%%%%%%%%%%%%%%%%%%%%%%%%%%%%%
\subsection{課題内容}
以下のプログラム {\tt report2-1.c} をコンパイルした結果をもとに, {\tt auto}変数と{\tt static}変数の違い,ポインタと配列の違いについてレポートせよ.

\subsection{与えられたプログラム}
\begin{verbatim}
      1	int primes_stat[10];
      2	
      3	char *string_ptr = "ABCDEFG";
      4	char string_ary[] = "ABCDEFG";
      5	
      6	void print_var(char *name, int val)
      7	{
      8	    print_string(name);
      9	    print_string(" = ");
     10	    print_int(val);
     11	    print_string("\n");
     12	}
     13	
     14	main()
     15	{
     16	    int primes_auto[10];
     17	
     18	    primes_stat[0] = 2;
     19	    primes_auto[0] = 3;
     20	
     21	    print_var("primes_stat[0]", primes_stat[0]);
     22	    print_var("primes_auto[0]", primes_auto[0]);
     23	}
\end{verbatim}

\subsection{与えられたプログラム(コンパイル後)}\label{sec:2-3}
\begin{verbatim}
      1	    .file	1 "2-3.c"
      2	
            (中略)
     13	
     14	
     15	    .rdata
     16	    .align	0
     17	    .align	2
     18	$LC0:
     19	    .ascii	"ABCDEFG\000"
     20	    .data
     21	    .align	0
     22	    .align	2
     23	_string_ptr:
     24	    .word	$LC0
     25	    .align	2
     26	_string_ary:
     27	    .ascii	"ABCDEFG\000"
     28	    .rdata
     29	    .align	0
     30	    .align	2
     31	$LC1:
     32	    .ascii	" = \000"
     33	    .align	2
     34	$LC2:
     35	    .ascii	"\n\000"
     36	    .text
     37	    .align	2
     38	    .set	nomips16
     39	_print_var:
     40	    subu	$sp,$sp,24
     41	    sw	$ra,20($sp)
     42	    sw	$fp,16($sp)
     43	    move	$fp,$sp
     44	    sw	$a0,24($fp)
     45	    sw	$a1,28($fp)
     46	    lw	$a0,24($fp)
     47	    jal	_print_string
     48	    la	$a0,$LC1
     49	    jal	_print_string
     50	    lw	$a0,28($fp)
     51	    jal	_print_int
     52	    la	$a0,$LC2
     53	    jal	_print_string
     54	    move	$sp,$fp
     55	    lw	$ra,20($sp)
     56	    lw	$fp,16($sp)
     57	    addu	$sp,$sp,24
     58	    j	$ra
     59	    .rdata
     60	    .align	0
     61	    .align	2
     62	$LC3:
     63	    .ascii	"primes_stat[0]\000"
     64	    .align	2
     65	$LC4:
     66	    .ascii	"primes_auto[0]\000"
     67	    .text
     68	    .align	2
     69	    .set	nomips16
     70	main:
     71	    subu	$sp,$sp,64
     72	    sw	$ra,60($sp)
     73	    sw	$fp,56($sp)
     74	    move	$fp,$sp
     75	    li	$v0,2			# 0x2
     76	    sw	$v0,_primes_stat
     77	    li	$v0,3			# 0x3
     78	    sw	$v0,16($fp)
     79	    la	$a0,$LC3
     80	    lw	$a1,_primes_stat
     81	    jal	_print_var
     82	    la	$a0,$LC4
     83	    lw	$a1,16($fp)
     84	    jal	_print_var
     85	    move	$sp,$fp
     86	    lw	$ra,60($sp)
     87	    lw	$fp,56($sp)
     88	    addu	$sp,$sp,64
     89	    j	$ra
     90	
     91	    .comm	_primes_stat,40
\end{verbatim}

\subsection{考察}

\subsubsection{{\tt auto}変数と{\tt static} 変数}

まず,{\tt auto}変数と{\tt static}変数の違いについて考察する.
{\tt auto}変数とは,関数内で動的に確保され,その関数が終了するまで値を保持し続ける変数である.今回与えられたプログラム内では{\tt primes\_auto}が該当する.
アセンブリコード内を見ると{\tt main}ラベルの直後で64バイトのスタックが確保されており,
{\tt primes\_auto}の値を保存するためにスタック領域が利用されているということが分かる.ここで,確保するスタックのサイズが64バイトとなっているのは,
手続き呼び出し規約で定められている最低限のスタックサイズ24バイトと{\tt primes\_auto}の値を格納するために必要な領域($4 × 10 = 40$バイト)の和が64バイトだからである.
これを確かめるために,簡易的なC言語のプログラムを作成した({\tt auto}変数に関する処理のみで構成されたプログラム).

\begin{verbatim}
      1	main()
      2	{
      3	    int primes_auto[2];
      4	
      5	    primes_auto[0] = 0;
      6	    primes_auto[1] = 1;
      7	
      8	    print_int(primes_auto[0]);
      9	    print_int(primes_auto[1]);
     10	}
\end{verbatim}

これをコンパイルすると以下のようなコードが生成される.

\begin{verbatim}
      1    		    .file	1 "2-3-2.c"
      2    	
                    (中略)
     13    	
     14    	
     15    		    .text
     16    		    .align	2
     17    	    main:
     18    		    subu	$sp,$sp,32        # 24 + 4 * 2
     19    		    sw	$ra,28($sp)
     20    		    sw	$fp,24($sp)
     21    		    move	$fp,$sp
     22    		    sw	$zero,16($fp)
     23    		    li	$v0,1			# 0x1
     24    		    sw	$v0,20($fp)
     25    		    lw	$a0,16($fp)
     26    		    jal	_print_int
     27    		    lw	$a0,20($fp)
     28    		    jal	_print_int
     29    		    move	$sp,$fp
     30    		    lw	$ra,28($sp)
     31    		    lw	$fp,24($sp)
     32    		    addu	$sp,$sp,32
     33    		    j	$ra
\end{verbatim}

このコンパイル結果からも分かるように,関数内で変数を宣言した場合は,関数の開始時に{\tt auto}変数の値を確保できるようにスタックを多めに確保している.
そして,変数の値はスタックに格納され,処理が終わるとともに値は失われてしまう.(スタックが解放される)

これに対して,{\tt static}変数とは,プログラムの開始時に静的に確保され,プログラムの終了時まで値を保持し続ける変数である.
今回与えられたプログラムでは{\tt primes\_stat}がこれに該当する.
アセンブリコード内を見ると,91行目のアセンブリ指令{\tt .comm}の部分でメモリ内に40バイト分の領域を確保しており,
{\tt primes\_stat}の値を保存するためにこの領域が使用されていることが分かる.
この領域はアセンブリ指令で確保された場所であり,プログラム終了まで解放されることはない(つまり,プログラムの終了時まで値を保持することができる).

この二種類の変数のメリットとデメリットについて比較した結果を表\ref{tbl:2-3}にまとめる.

\begin{table}[]
      \centering
      \caption{二種類の変数の比較}
      \label{tbl:2-3}
      \begin{tabular}{|l|l|l|}
            \hline 
                  & {\tt auto変数}                                                                                                                                                    & {\tt static変数}                                                                                                                     \\ \hline
            保存場所  & メモリ(スタック部)                                                                                                                                                & メモリ(データ部)                                                                                                                    \\ \hline
            メリット  & \begin{tabular}[c]{@{}l@{}}関数の中でのみ値を保持しているので\\ メモリ空間の節約ができる.\\ (使うときだけ確保する)\\ スタックに積み上げながら\\値を保持している ので,再帰呼び出しで\\書き込みが起きた際でも,\\別々の場所に値を格納できる.\end{tabular} & \begin{tabular}[c]{@{}l@{}}メモリ内の固定した位置に値を\\ 格納しているので,再利用が容易.\\ スタックを使う場合と比較して,\\ 読み書きを素早く行うことができる.\\ (確保,解放が不要)\end{tabular} \\ \hline
            デメリット & \begin{tabular}[c]{@{}l@{}}同じ値を何度も呼び出す場合だと,\\ スタックの確保および解放処理を\\ 無駄に実行する事になる.\end{tabular}                                                                 & \begin{tabular}[c]{@{}l@{}}値を格納する位置は一箇所に固定\\ されているため,再帰呼び出しを\\ してしまうと意図せず値を\\ 上書きしてしまう危険性がある.\end{tabular}                   \\ \hline
            \end{tabular}
      \end{table}

\subsubsection{配列とポインタ}

次に,配列とポインタの違いについて考察する.C言語でこれらを用いる際には,変数に格納されているのはどちらも先頭アドレスでなので,
ほぼ同様に扱えるが,コンパイル後のアセンブリコードを見ると,明確な相違点が存在する.
それは,配列として宣言している場合は該当ラベルの位置に値がそのまま書かれている(\ref{sec:2-3}節のプログラムの27行目)のに対し,
ポインタとして宣言している場合は該当ラベルの位置に直接値が書かれているのではなく,その値が書かれているラベルが書かれており(\ref{sec:2-3}節のプログラムの24行目),
間接的に値にアクセスしているということである.

この二つを比較するために作成した簡易的なC言語のプログラムを以下に示す.

\begin{verbatim}
      1  char *string_ptr = "ABCDEFG";
      2  char string_ary[] = "ABCDEFG";
      3
      4  main()
      5  {
      6      string_ptr = "AB";
      7      //string_ary = "AB"; コンパイルエラー発生
      8  }
\end{verbatim}

これをコンパイルすると以下のようなコードが生成される.

\begin{verbatim}
      1          .file   1 "2-3-3.c"
      2
                  (中略)
     13
     14
     15          .rdata
     16          .align  2
     17  $LC0:
     18          .asciiz "ABCDEFG"
     19          .data
     20          .align  2
     21  _string_ptr:
     22          .word   $LC0
     23          .align  2
     24  _string_ary:
     25          .asciiz "ABCDEFG"
     26          .rdata
     27          .align  2
     28  $LC1:
     29          .asciiz "AB"
     30          .text
     31          .align  2
     32  main:
     33          subu    $sp,$sp,8
     34          sw      $fp,0($sp)
     35          move    $fp,$sp
     36          la      $v0,$LC1
     37          sw      $v0,_string_ptr
     38          move    $sp,$fp
     39          lw      $fp,0($sp)
     40          addu    $sp,$sp,8
     41          j       $ra
\end{verbatim}

上記のプログラムから分かるように,配列とポインタの大きな違いとしては,宣言した変数に再び代入処理ができるかどうか,ということが挙げられる.
配列の場合では,変数内に文字列の先頭アドレスが定数として保存されているので,別の値を代入することはできずコンパイルエラーが発生する.(添字付きでアクセスするのは問題ない)
これに対してポインタの場合では,変数内に格納されているアドレスを自由に付け替えれる様になっているので,再代入が可能であり,それによって元の値が壊されることもない.
(アクセスする場所が変わっているだけ).ただ,ポインタを使った場合では使わなくなった値を適切に解放する処理をしなければ,
どこからもアクセスされないデータが蓄積し,メモリを圧迫する場合もあるので,プログラマ側が注意する必要がある.

%%%%%%%%%%%%%%%%%%%%%%%%%%%%%%%%%%%%%%%%%%%%%%%%%%%%%%%%%%%%%%%%
\section{課題2-4}
%%%%%%%%%%%%%%%%%%%%%%%%%%%%%%%%%%%%%%%%%%%%%%%%%%%%%%%%%%%%%%%%
\subsection{課題内容}

{\tt printf} など,一部の関数は,任意の数の引数を取ることができる.
これらの関数を可変引数関数と呼ぶ.
MIPSのCコンパイラにおいて可変引数関数の実現方法について考察し,解説せよ.

\subsection{可変引数とは}

可変引数とは,引数の数を呼び出す側が任意で変更することができる関数である.
例えば,{\tt printf}関数を実装する際に,任意の数の引数に対応している場合,
引数の数が2つで固定されてしまった場合と比較して,関数を呼び出す回数を減らすことができる,というメリットがある.
つまり,何か変数の値を表示したい時に,表示したい個数分{\tt printf}を呼び出す必要がなくなる.

\subsection{可変引数の実現}

C言語で可変引数を扱う際には{\tt void printf(char *fmt,...)}の様な形式で関数を宣言する.
この形式で記述された関数がアセンブリ言語の中でどの様に実現されているのかについて以下に解説する.
アセンブリでは可変引数を利用する際に,スタック領域を活用している.課題2-2の表\ref{tbl:2-2}でも示している様に,
呼び出し側が余分に取っておいたスタック領域に順番に引数を格納しており,もし引数の数が4個より多くなったとしても,
それに応じてスタックの確保領域を増やし,第4引数の次の領域から順に格納していくことで対応する.(関数内で引数に順番にアクセスしやすくするため)
この方法を用いることで,スタック領域が埋まらない限りは,何個でも引数の数を増やすことができる.

次に,与えられたそれぞれの引数に対して,どの様にアクセスしているかについて解説する.
スタックを用いて引数を順番に確保しているとはいえ,引数の型がそれぞれ異なる場合があるので,
アドレスをいくつ加算するかが不明な場合は,正しくデータにアクセスすることができない.
そのため,可変引数を用いる際には{\tt sizeof}演算子を利用し,与えられた引数に応じて次の引数にアクセスするために,
アドレスをいくつ加算すれば良いかを計算している.(例えば,{\tt Int}型は4バイトであり,{\tt char}型は1バイト)
ちなみに,この{\tt sizeof}演算子は32ビットCPUと64ビットCPUで異なる挙動を示すので,同じソースコードでも動かす環境次第で加算する数を変えなければならない.

\subsection{アセンブリでの実装}

可変引数についてプログラムを元に説明するため,簡易的なプログラムを作成した.

\begin{verbatim}
      1	void fun(int a, ...)
      2	{
      3	    return;
      4	}
      5	
      6	int main()
      7	{
      8	
      9	    fun(1, 1, 'a', "a", 2, 'b', "b");
     10	
     11	    return 0;
     12	}      
\end{verbatim}

これをコンパイルすると以下のようなコードが生成される.

\begin{verbatim}
      1		.file	1 "2-4.c"
      2	
                  (中略)
     13	
     14	
     15		.text
     16		.align	2
     17	_fun:
     18		sw	$a0,0($sp)
     19		sw	$a1,4($sp)
     20		sw	$a2,8($sp)
     21		sw	$a3,12($sp)
     22		subu	$sp,$sp,8
     23		sw	$fp,0($sp)
     24		move	$fp,$sp
     25		sw	$a0,8($fp)
     26		move	$sp,$fp
     27		lw	$fp,0($sp)
     28		addu	$sp,$sp,8
     29		j	$ra
     30		.rdata
     31		.align	2
     32	$LC0:
     33		.asciiz	"a"
     34		.align	2
     35	$LC1:
     36		.asciiz	"b"
     37		.text
     38		.align	2
     39	main:
     40		subu	$sp,$sp,40
     41		sw	$ra,36($sp)
     42		sw	$fp,32($sp)
     43		move	$fp,$sp
     44		li	$v0,2			# 0x2
     45		sw	$v0,16($sp)
     46		li	$v0,98			# 0x62
     47		sw	$v0,20($sp)
     48		la	$v0,$LC1
     49		sw	$v0,24($sp)
     50		li	$a0,1			# 0x1
     51		li	$a1,1			# 0x1
     52		li	$a2,97			# 0x61
     53		la	$a3,$LC0
     54		jal	_fun
     55		move	$v0,$zero
     56		move	$sp,$fp
     57		lw	$ra,36($sp)
     58		lw	$fp,32($sp)
     59		addu	$sp,$sp,40
     60		j	$ra       
\end{verbatim}

ここで注目すべきは40-53行目である.この処理の中でのスタック内の要素について図示すると表\ref{tbl:2-4}の様になり,
先ほど説明した様に,スタックを必要に応じて拡張することで,可変引数に対応することが可能となっている,ということが分かる.
また,手続き呼び出し規約のために,時系列的には第1引数〜第4引数よりも先に第5引数〜第7引数がスタックに格納されているが,
第4引数の次のアドレスに第5引数が格納される様になっているので,引数が並び替えられていいるわけではない.


\begin{table}[]
      \centering
      \caption{スタックを用いた可変引数の保存}
      \label{tbl:2-4}
      \begin{tabular}{|l|l|l|}
      \hline
      offset & 内容   & 備考   \\ \hline
      -40    & \$a0 & 第1引数 \\ \hline
      -36    & \$a1 & 第2引数 \\ \hline
      -32    & \$a2 & 第3引数 \\ \hline
      -28    & \$a3 & 第4引数 \\ \hline
      -24    & \$v0 & 第5引数 \\ \hline
      -20    & \$v0 & 第6引数 \\ \hline
      -16    & \$v0 & 第7引数 \\ \hline
      -12    &      &      \\ \hline
      -8     & \$fp & フレームポインタ \\ \hline
      -4     & \$ra & 戻りアドレス \\ \hline
      \end{tabular}
      \end{table}

%%%%%%%%%%%%%%%%%%%%%%%%%%%%%%%%%%%%%%%%%%%%%%%%%%%%%%%%%%%%%%%%
\section{課題2-5}
%%%%%%%%%%%%%%%%%%%%%%%%%%%%%%%%%%%%%%%%%%%%%%%%%%%%%%%%%%%%%%%%

\subsection{課題内容}

{\tt printf} のサブセットを実装し, SPIM上でその動作を確認する応用プログラム(自由なデモプログラム)を作成せよ. 
フルセットにどれだけ近いか,あるいは,よく使う重要な仕様だけをうまく切り出して, 
実用的なサブセットを実装しているかについて評価する. 
ただし,浮動小数は対応しなくてもよい(SPIM自体がうまく対応していない). 
加えて,この {\tt printf} を利用した応用プログラムの出来も評価の対象とする.

%%%%%%%%%%%%%%%%%%%%%%%%%%%%%%%%%%%%%%%%%%%%%%%%%%%%%%%%%%%%%%%%

\subsection{プログラムの作成方針}
今回のプログラムを作成するに当たって,以下の部分から構成することにした.

\begin{itemize}
      \item 作成したサブセットそれぞれをテストする部分
      \item 与えられたフォーマット指定子を元に,引数を適切に出力する部分
      \item 表示桁数や0埋めなどの指定を元に,要素を修飾していく部分
\end{itemize}

\subsection{プログラムおよびその説明}

以下では,前節の作成方針における分類に基づいて,プログラムの大まかな構造について解説する.
また,プログラムのソースコードについては\ref{sec:sourcecode}節に添付している.

\subsubsection{テスト部(5行目から27行目)}
この部分では,表\ref{tbl:2-5}に示された,今回実装したサブセットについて,網羅的にテストを行っている.

\begin{table}[]
      \centering
      \caption{実装したサブセット}
      \label{tbl:2-5}
      \begin{tabular}{|l|l|l|l|}
            \hline
            サブセット & 種別 & 型 & 内容                  \\ \hline
            \%c   & 変換指定子 & char  & 任意の一文字を出力           \\ \hline
            \%s   & 変換指定子 & char* & 任意の文字列を出力           \\ \hline
            \%d   & 変換指定子 & int   & 任意の数値を出力(10進数表記)    \\ \hline
            \%o   & 変換指定子 & int   & 任意の数値を出力(8進数表記)     \\ \hline
            \%x   & 変換指定子 & int   & 任意の整数を出力(16進数表記)    \\ \hline
            \%-  & フラグ & 無し    & 左詰め           \\ \hline
            \%0  & フラグ & 無し    & 0埋め             \\ \hline
            \%+  & フラグ & 無し    & 符号を表示         \\ \hline
            \%n(任意の整数)  & その他 & 無し    & n文字表示(最大文字数)   \\ \hline
            \%.n (任意の整数) & その他 & 無し    & 表示文字数の上限を指定(\%sでのみ動作)   \\ \hline
            \%\%  & その他 & 無し    & "\%"を出力             \\ \hline
      \end{tabular}
\end{table}

\subsubsection{引数出力部(157行目から320行目)}

この部分では,与えられた文字列{{\tt fmt}}を解析し,\%があれば,後述する要素修飾部を実行した後,
表\ref{tbl:2-5}の変換指定子を元にどの型が可変引数として与えられたかを推定し,システムコールを用いて値を出力する.

\subsubsection{要素修飾部(172行目から203行目)}

この部分では{{\tt fmt}}を解析し,\%を発見した後,表\ref{tbl:2-5}のフラグや表示桁数を読み取り,出力する文字を適切に装飾する.

\subsection{実行結果}
\begin{verbatim}
one character [a]
string [hello]
decimal 100 = [100]
octal 100 = [144]
hexadecimal 100 = [64]
escape % [%]
max 10 characters [     hello]
limit two characters [he]
max 5 digits decimal [  100]
max 5 digits octal [  144]
max 5 digits hexadecimal [   64]
zero padding decimal [00100]
zero padding octal [00144]
zero padding hexadecimal [00064]
left decimal [100  ]
left octal [144  ]
left hexadecimal [64   ]
sign [+100]
\end{verbatim}

\subsection{考察}
課題のプログラムを作成する上で,まず最初に気をつけた点は可変引数の取り扱いである.前の課題で述べた様に,
与えられた引数の型に応じて加算するアドレスの大きさを変えなければならないので,{{\tt switch}}文で分岐し,
出力処理が終わったタイミングでアドレスの加算処理を行う様にした(分岐するまでは,想定されている引数の型が不明なので).

また,8進数表記や16進数に対応するために,再帰呼び出しに対応した関数を作成した.
なぜ再帰呼び出しにする必要があったのかという理由については,
進数を変換する過程で最後に出力すべき数字から順に値が求まる様になっているからである.

この関数の処理の流れは以下の通りである.
\begin{enumerate}
      \item 引数として,任意の数字({\tt num})および基数{\tt base}を受け取る.
      \item ${\tt num}\div{\tt base}$を計算し,商と余りを求める.
      \item 商が$0$でなければ,その商と基数を引数として,再帰的に関数を呼び出す.
      \item 商が$0$になるまで上記の手順を繰り返す.(スタックに表示待ちの数字を積み上げている)
      \item 商が$0$になると,スタックに積み上げられた数字を順に取り出し出力する.
      \item 出力する際,$10$以上の数字はアルファベットに変換して表示する.
\end{enumerate}

この関数は,引数として基数も受け取れるようにしているので,8進数や16進数以外の表示することも可能である.(今回のプログラムでは$10 + 26$で36進数まで対応可能)


\subsection{作成したプログラム}\label{sec:sourcecode}
\begin{verbatim}
      1	#define ROUNDUP_SIZEOF(a) ((sizeof(a) + 3) / 4 * 4)
      2	
      3	void myprintf(char *fmt, ...);
      4	
      5	int main()
      6	{
      7	    myprintf("one character [%c]", 'a');
      8	    myprintf("string [%s]", "hello");
      9	    myprintf("decimal 100 = [%d]", 100);
     10	    myprintf("octal 100 = [%o]", 100);
     11	    myprintf("hexadecimal 100 = [%x]", 100);
     12	    myprintf("escape %% [%%]");
     13	    myprintf("max 10 characters [%10s]", "hello");
     14	    myprintf("limit two characters [%.2s]", "hello");
     15	    myprintf("max 5 digits decimal [%5d]", 100);
     16	    myprintf("max 5 digits octal [%5o]", 100);
     17	    myprintf("max 5 digits hexadecimal [%5x]", 100);
     18	    myprintf("zero padding decimal [%05d]", 100);
     19	    myprintf("zero padding octal [%05o]", 100);
     20	    myprintf("zero padding hexadecimal [%05x]", 100);
     21	    myprintf("left decimal [%-5d]", 100);
     22	    myprintf("left octal [%-5o]", 100);
     23	    myprintf("left hexadecimal [%-5x]", 100);
     24	    myprintf("sign [%+d]", 100);
     25	
     26	    return 1;
     27	}
     28	
     29	int max(int a, int b)
     30	{
     31	    if (a > b)
     32	    {
     33	        return a;
     34	    }
     35	    else
     36	    {
     37	        return b;
     38	    }
     39	}
     40	
     41	int min(int a, int b)
     42	{
     43	    if (a < b)
     44	    {
     45	        return a;
     46	    }
     47	    else
     48	    {
     49	        return b;
     50	    }
     51	}
     52	
     53	int get_digit(int a, int base)
     54	{
     55	    int digit = 0;
     56	
     57	    if (a == 0)
     58	    {
     59	        return 0;
     60	    }
     61	    else
     62	    {
     63	        while (a != 0)
     64	        {
     65	            a /= base;
     66	            digit++;
     67	        }
     68	        return digit;
     69	    }
     70	}
     71	
     72	int get_range(char *str)
     73	{
     74	    int range = 0;
     75	
     76	    while ((*str >= '0' && *str <= '9'))
     77	    {
     78	        range = range * 10 + (*str - '0');
     79	        str++;
     80	    }
     81	    return range;
     82	}
     83	
     84	char get_char_for_fill(int is_zero)
     85	{
     86	    if (is_zero)
     87	    {
     88	        return '0';
     89	    }
     90	    else
     91	    {
     92	        return ' ';
     93	    }
     94	}
     95	
     96	void print_plus(int is_plus)
     97	{
     98	    if (is_plus)
     99	        print_char('+');
    100	}
    101	
    102	void print_fill(int size, int is_zero)
    103	{
    104	    int i;
    105	    for (i = 0; i < size; i++)
    106	    {
    107	        print_char(get_char_for_fill(is_zero));
    108	    }
    109	}
    110	
    111	void print_limited_string(int limit, char *str)
    112	{
    113	    int i;
    114	    for (i = 0; i < limit; i++)
    115	    {
    116	        if (*str == '\0')
    117	        {
    118	            break;
    119	        }
    120	        print_char(*str++);
    121	    }
    122	}
    123	
    124	int strlen(char *str)
    125	{
    126	    int length = 0;
    127	
    128	    while (*str++ != '\0')
    129	    {
    130	        length++;
    131	    }
    132	
    133	    return length;
    134	}
    135	
    136	void print_base(int num, int base)
    137	{
    138	    int surplus = num % base;
    139	    int quotient = num / base;
    140	
    141	    if (quotient != 0)
    142	    {
    143	        print_base(quotient, base);
    144	    }
    145	
    146	    if (surplus >= 10)
    147	    {
    148	        print_char('a' + surplus - 10);
    149	    }
    150	    else
    151	    {
    152	        print_int(surplus);
    153	    }
    154	    return;
    155	}
    156	
    157	void myprintf(char *fmt, ...)
    158	{
    159	    int leftrange = 0;
    160	    int rightrange = 0;
    161	    int read_zero = 0;
    162	    int is_left = 0;
    163	    int is_plus = 0;
    164	
    165	    char *p = (char *)&fmt + ROUNDUP_SIZEOF(fmt);
    166	    while (*fmt)
    167	    {
    168	        if (*fmt == '%')
    169	        {
    170	            fmt++;
    171	
    172	            leftrange = 0;
    173	            rightrange = 0;
    174	            read_zero = 0;
    175	            is_left = 0;
    176	            is_plus = 0;
    177	
    178	            while (*fmt == '0' || *fmt == '-' || *fmt == '+')
    179	            {
    180	                switch (*fmt)
    181	                {
    182	                case '0':
    183	                    read_zero = 1;
    184	                    break;
    185	                case '-':
    186	                    is_left = 1;
    187	                    break;
    188	                case '+':
    189	                    is_plus = 1;
    190	                    break;
    191	                default:
    192	                    break;
    193	                }
    194	                fmt++;
    195	            }
    196	
    197	            leftrange = get_range(fmt);
    198	            fmt += get_digit(leftrange, 10);
    199	            if (*fmt == '.')
    200	            {
    201	                rightrange = get_range(++fmt);
    202	                fmt += get_digit(rightrange, 10);
    203	            }
    204	
    205	            switch (*fmt)
    206	            {
    207	            case 'c':
    208	                print_char(*(char *)p);
    209	                p += ROUNDUP_SIZEOF(char);
    210	                break;
    211	            case 's':
    212	
    213	                if (rightrange == 0)
    214	                {
    215	                    rightrange = strlen(*(char **)p);
    216	                }
    217	
    218	                if (is_left == 1)
    219	                {
    220	                    print_limited_string(rightrange, *(char **)p);
    221	                    print_fill(
    222	                        max(leftrange - strlen(*(char **)p), 0), 0);
    223	                }
    224	                else
    225	                {
    226	                    print_fill(
    227	                        max(leftrange - strlen(*(char **)p), 0), 0);
    228	                    print_limited_string(rightrange, *(char **)p);
    229	                }
    230	                p += ROUNDUP_SIZEOF(char *);
    231	                break;
    232	            case 'd':
    233	                if (is_left == 1)
    234	                {
    235	                    print_plus(is_plus);
    236	                    print_int(*(int *)p);
    237	                    print_fill(
    238	                        max(
    239	                            leftrange -
    240	                                get_digit(*(int *)p, 10) - is_plus,
    241	                            0),
    242	                        0);
    243	                }
    244	                else
    245	                {
    246	                    print_fill(
    247	                        max(
    248	                            leftrange -
    249	                                get_digit(*(int *)p, 10) - is_plus,
    250	                            0),
    251	                        read_zero == 1);
    252	                    print_plus(is_plus);
    253	                    print_int(*(int *)p);
    254	                }
    255	                p += ROUNDUP_SIZEOF(int);
    256	                break;
    257	            case 'o':
    258	                if (is_left == 1)
    259	                {
    260	                    print_plus(is_plus);
    261	                    print_base(*(int *)p, 8);
    262	                    print_fill(
    263	                        max(
    264	                            leftrange -
    265	                                get_digit(*(int *)p, 8) - is_plus,
    266	                            0),
    267	                        0);
    268	                }
    269	                else
    270	                {
    271	                    print_fill(
    272	                        max(
    273	                            leftrange -
    274	                                get_digit(*(int *)p, 8) - is_plus,
    275	                            0),
    276	                        read_zero == 1);
    277	                    print_plus(is_plus);
    278	                    print_base(*(int *)p, 8);
    279	                }
    280	                p += ROUNDUP_SIZEOF(int);
    281	                break;
    282	            case 'x':
    283	                if (is_left == 1)
    284	                {
    285	                    print_plus(is_plus);
    286	                    print_base(*(int *)p, 16);
    287	                    print_fill(
    288	                        max(
    289	                            leftrange -
    290	                                get_digit(*(int *)p, 16) - is_plus,
    291	                            0),
    292	                        0);
    293	                }
    294	                else
    295	                {
    296	                    print_fill(
    297	                        max(
    298	                            leftrange -
    299	                                get_digit(*(int *)p, 16) - is_plus,
    300	                            0),
    301	                        read_zero == 1);
    302	                    print_plus(is_plus);
    303	                    print_base(*(int *)p, 16);
    304	                }
    305	                p += ROUNDUP_SIZEOF(int);
    306	                break;
    307	            case '%':
    308	                print_char('%');
    309	                break;
    310	            default:
    311	                break;
    312	            }
    313	        }
    314	        else
    315	        {
    316	            print_char(*fmt);
    317	        }
    318	        fmt++;
    319	    }
    320	    print_char('\n');
    321	    return;
    322	} 
\end{verbatim}



\section{感想}
今回の課題を進めていく中で,システムコールの実態や{\tt printf}の内部実装等,
今までブラックボックスだった部分の内部を詳しく知ることができた.
今後エンジニアとして生きていく上で,この様な内部実装を詳しく知っているのといないのとでは,
プログラムでエラーが発生した時の解決スピードに大きな差が出てくると思うので,
「とりあえず動くから大丈夫」で終わるのではなく「実際にプログラムがどの様な処理を実行しているのか」を,
ライブラリの内部実装などを含めてきちんと理解しながらプログラムを実装していくのが重要だなと思った.

\end{document}
